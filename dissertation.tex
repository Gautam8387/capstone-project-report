%-------------------------------------------------------------------------------
%
% TUM Dissertation Template
%
% For usage instructions see README.md
%
% Authors:
%   Andre Richter, andre.richter@tum.de
%   Michael Vonbun, michael.vonbun@tum.de
%   Christian Herber, christian.herber@tum.de
%   Stefan Wallentowitz, stefan.wallentowitz@tum.de
%
%-------------------------------------------------------------------------------
\documentclass[%
  % layouttitlepage,            % layout help rules (to see if you need
  %                             % some extra vspace in your title etc.)
  % headings = standardclasses, % serif fonts for headings
  % headings = big,             % If you use serif fonts for headings (above option
  %                             % uncommmented), uncomment this one to get smaller
  %                             % headings
  % sansseriftitlepage,         % sans serif title page
  % notocintoc,                 % do not add toc to toc itself
]{tumDiss}
\usepackage[utf8]{inputenc}

\usepackage[left=30mm,right=30mm,top=30mm,bottom=30mm]{geometry}
\usepackage{fancyhdr}
\usepackage{mdframed}
\usepackage{lettrine}
\usepackage{float}
\usepackage{hyperref}
\hypersetup{
    colorlinks=true,
    linkcolor=blue,
    filecolor=magenta,      
    urlcolor=cyan,
}

\pagestyle{fancy}
\fancyhf{} % clear all header and footer fields
\fancyhead[L]{Ahuja, Gautam} % title on left
\fancyhead[R]{\rightmark} % section on right
\fancyfoot[C]{\thepage} % page number in center
\renewcommand{\headrulewidth}{0.4pt}

%-------------------------------------------------------------------------------
% Binding correction for the title page.
% WARNING: ONLY NEEDED FOR THE PRINT VERSION!
%
% After printing and binding, the left part of the titlepage may lose
% significant space, for example due to overlap from glue binding.
% You can increase the left margin of the title page with this option.
% This value of 8mm was measured for glue binding a thesis that was printed by
% the TUM Fachschaft EI and ~140 pages.
%-------------------------------------------------------------------------------
% \titlepagebindingcor{8mm}

%-------------------------------------------------------------------------------
% Binding correction for everything else.
% Does not affect titlePageBindingCor!
%
% WARNING: THIS OPTION CAN SHAKE UP YOUR CURRENT LAYOUT.
% If you want to use it, it is best to work with it from the very start. Adding
% it when finishing your dissertation might get you into trouble.
%
% Search http://texdoc.net/texmf-dist/doc/latex/koma-script/scrguien.pdf for
% "BCOR" for further reading.
%-------------------------------------------------------------------------------
% \KOMAoptions{BCOR=3mm}



%-------------------------------------------------------------------------------
% Faculty
%-------------------------------------------------------------------------------
\faculty{Department of Computer Science}

%-------------------------------------------------------------------------------
% Degree
%-------------------------------------------------------------------------------
% \degree{Doktor-Ingenieurs (Dr.-Ing.)}
\degree{Postgraduate Diploma in Advanced Studies and Research (DipASR)}

%-------------------------------------------------------------------------------
% Title
%
% IMPORTANT:
%
% You must add manual line breaks here. If you don't, you'll get uneven spacing
% between the lines.
% YOU ALSO NEED THE BREAK AT THE LAST LINE.
%-------------------------------------------------------------------------------
\title{%
  Linguistic-agnostic Intelligent Support Systems for\\
   Probable Diagnosis to Aid in Clinical Settings\\
}
% \subtitle{--Extended Titile--}

%-------------------------------------------------------------------------------
% People
%-------------------------------------------------------------------------------
\author{Gautam Ahuja}
\vorsitz{Dr. Anurag Agarwal, Dean, Trivedi School of BioSciences, Ashoka University}
\vorsitzz{Dr. Govind K Makharia, Professor, Department of Gastroenterology and Human Nutrition Adjunct Faculty, AIIMS - New Delhi}
\vorsitzzz{Dr. Rintu Kutum, Faculty Fellow, Department of Computer Science, Ashoka University}
\erstpruef{Dr. Partha Pratim Das, Professor of Computer Science, Ashoka University}

% Use this one for a TUM professor
% \zweitpruef{Prof. Dr.-Ing. Vorname Nachname}

% Or this one for an external professor
\zweitpruef[AIIMS, New Delhi]{Dr. Ayush Agarwal, Department of Gastroenterology and Human Nutrition}

% Optionally, add a third examiner
%\drittpruef[RWTH Aachen]{Prof. Dr.-Ing. Max Mustermann}

%-------------------------------------------------------------------------------
% Hand in date
%
% This is the date of your personal hand-in at the TUM doctoral office.
%-------------------------------------------------------------------------------
\date{01.01.2016}

%-------------------------------------------------------------------------------
% Accepted date
%
% You can set this after your thesis was accepted. For handing in,
% it is not needed (at least for the Electrical Engineering faculty).
%-------------------------------------------------------------------------------
\dateaccepted{10.05.2016}



%-------------------------------------------------------------------------------
% Change language, e.g. to german
%-------------------------------------------------------------------------------
% \usepackage[ngerman]{babel}

%-------------------------------------------------------------------------------
% Compatibility issues
%-------------------------------------------------------------------------------
% If you need pstricks, load it here before everything else.
% Otherwise, tikz patterns won't work
%\usepackage{pstricks}

%-------------------------------------------------------------------------------
% Suggested standard packets are included here
%-------------------------------------------------------------------------------
% Loading scrhack fixes:
%   (1) KOMA-Script incompatible macros used in listings package.
%   (2) Inconsistent anchors in hyperref.
\usepackage{scrhack}


% figure inclusion
\usepackage[
  caption = false,
  font    = footnotesize
]{subfig}
\usepackage{graphicx}
\usepackage{pgfplots}
\usepackage{pgfplotstable}
\tikzset{>=stealth}
\usetikzlibrary{patterns}
\usetikzlibrary{pgfplots.statistics}

% code block insertion
\usepackage{moreverb}
\usepackage{listings}

% math and equations
\usepackage{amsmath}
\usepackage{amssymb}
\usepackage{amsfonts}
\usepackage{upgreek}

% enumeration
\usepackage{enumerate}

% Source code with highlighting
\usepackage{listings}
\lstset{
  basicstyle       = \footnotesize,
  captionpos       = b,
  tabsize          = 4,
  commentstyle     = \color{TUMGreen},
  keywordstyle     = \color{TUMBlue},
  stringstyle      = \color{TUMOrange},
  otherkeywords    = {
    uint64_t,
    uint32_t,
    uint16_t,
    uint8_t,
    u64,
    u32,
    u16,
    u8,
    inline
  },
  numbers          = left,
  xleftmargin      = 7ex,
  aboveskip        = 4ex,
  abovecaptionskip = 2ex,
}

% Support for siunitx
\usepackage{siunitx}
\sisetup{
  exponent-product = \cdot,
  output-product   = \cdot,
  per-mode         = symbol-or-fraction,
  quotient-mode    = fraction,
  binary-units     = true
}

% No widows and orphans
\usepackage[all]{nowidow}

% dummy text
\usepackage{lipsum}

% hyperlinks
% according to its documentation, hyperref should be loaded last
% a list of packages that should be loaded after hyperref can be found at
% https://tex.stackexchange.com/questions/1863/which-packages-should-be-loaded-after-hyperref-instead-of-before
\usepackage{url}
\usepackage[
  hidelinks,
  bookmarksnumbered
]{hyperref}

% If hyperref is used, references to tables and figures link to their captions
% and not the actual tables or figures. This is especially unwanted for figures,
% because their captions are below the figure so that clicking on a link just
% shows the captions and the figure is invisible.
%
% Using the caption package fixes this behaviour.
\usepackage{caption}

% glossary functionality
% loading glossaries after hyperref adds hyperlings to acronyms and glossary
% entries
\usepackage[
  toc,
  acronym,
  style = long
]{glossaries}
\makeglossaries

%-------------------------------------------------------------------------------
% Include custom packages here
%-------------------------------------------------------------------------------

% \usepackage{}


%-------------------------------------------------------------------------------
% TUM CI colors for PGF
%-------------------------------------------------------------------------------
\definecolor{grey60} {RGB} {102, 102, 102} % 60% grey

%-------------------------------------------------------------------------------
% Default values for pgfplots
%-------------------------------------------------------------------------------
\newcommand{\figureHeight}{0.5625} %% 16:9
\pgfplotsset{
  compat           = 1.13,
  grid             = major,
  enlarge x limits = 0,
  cycle list name  = tum,
  major grid style = {dotted},
  minor grid style = {dotted},
  legend style     = {
    at     = {(0.98,0.96)},
    anchor = north east,
  },
  width            = \hsize * 0.9,
  height           = \hsize * 0.9 * \figureHeight,
}

%-------------------------------------------------------------------------------
% Correct bad hyphenation here
%-------------------------------------------------------------------------------
\hyphenation{op-tical net-works semi-conduc-tor}

%-------------------------------------------------------------------------------
% Acronyms (will be sorted alphabetically)
%-------------------------------------------------------------------------------
\newacronym{osllm}{osLLM}{Open Source Large Language Model}
\glsadd{osllm}

\newacronym{llm}{LLM}{Large Language Model}
\glsadd{llm}

\newacronym{ai}{AI}{Artificial Intelligence}
\glsadd{ai}

\newacronym{nlp}{NLP}{Natural Language Processing}
\glsadd{nlp}

\newacronym{cdss}{CDSS}{Clinical Decision Support System}
\glsadd{cdss}

\newacronym{cds}{CDS}{Clinical Decision Support}
\glsadd{cds}

\newacronym{aim}{AIM}{Artificial Intelligence in Medicine}
\glsadd{aim}

\newacronym{opd}{OPD}{Outpatient Department}
\glsadd{opd}

\newacronym{aiims}{AIIMS}{All India Institute of Medical Sciences}
\glsadd{aiims}



%-------------------------------------------------------------------------------
% Actual document starts here
%-------------------------------------------------------------------------------
\begin{document}
\frontmatter
\maketitle


%-------------------------------------------------------------------------------
\chapter{Acknowledgment}
I want to thank my advisor, Dr. Rintu Kutum. I am grateful for his guidance, support, and encouragement throughout my project. I would also like to thank Dr. Anurag Agarwal and Dr. Govind K Makharia for their valuable suggestions and feedback on my project. I want to acknowledge the help of Sanjana Ahuja, who helped me with the initial design of my project, bringing in her expertise in the field of creative design. I want to thank my colleagues at the Augmented Health Systems Lab for their support and feedback. I want to thank my family and friends for their constant support and encouragement. Lastly, I would like to thank the Koita Centre for Digital Health - Ashoka, Department of Computer Science, Ashoka University, and AIIMS, New Delhi, for providing me with the necessary guidelines, resources and infrastructure to carry out my project.


%-------------------------------------------------------------------------------
\chapter{Certificate}
% certificat.tex
I, \textbf{Gautam Ahuja}, declare that the work entitled \textbf{Linguistic-agnostic Intelligent Support Systems for probable diagnosis to aid in Clinical Settings} is an authentic record of my work carried out at the \textbf{Department of Computer Science, Ashoka University}, under the guidance of my supervisor, \textbf{Dr. Rintu Kutum} for the fulfillment of the requirements of the degree of \textbf{Postgraduate Diploma in Advanced Studies and Research (DipASR)} in the \textbf{Department of Computer Science} at \textbf{Ashoka University}. \\% [\baselineskip]

\noindent I further declare that this thesis has not been submitted, in part or in full, for the award of any other degree, diploma, or certificate at this or any other institution. All sources of information, ideas, or quotations that are not my own have been duly acknowledged in the bibliography. \\ % [\baselineskip]


\noindent This declaration is made with the full knowledge and understanding of the university's regulations on academic honesty and integrity. \\% [\baselineskip]


\noindent This thesis has been approved by the thesis advisor and the examination committee, and it fulfills the requirements for the award of the degree.\\% [\baselineskip]

\begin{flushleft}
\textbf{Date:} December 08, 2024
\end{flushleft}

\begin{flushright}
\includegraphics[width=0.3\textwidth]{images/signature-ga.jpeg}\\
\textbf{Gautam Ahuja}\\
ASP' 25, 1020211395\\
Department of Computer Science\\
Ashoka University\\
\end{flushright}

% Advisor
\vspace{1.25cm}
\begin{flushright}
\textbf{Dr. Rintu Kutum}\\
Department of Computer Science\\
Ashoka University\\
\end{flushright}


%-------------------------------------------------------------------------------
% \chapter{Abstract}
% % This file is ./abstract.tex
% Citation file is ./references.bib


Using Input
\lipsum[1-4]
This is sample text with a citation \cite{barham2003xen}. Multiple citations \cite{barham2003xen,LIS} can be combined.


%-------------------------------------------------------------------------------
\tableofcontents
\listoffigures
\listoftables
\printglossary[type=\acronymtype, nonumberlist]


%-------------------------------------------------------------------------------
\mainmatter
\chapter{Introduction}
\label{chap:introduction}
\lettrine{C}{ }omputer-aided clinical decision-making and reasoning have long been considered a model of human behavior. For many years, it has influenced and been the subject of \gls{ai} study \cite{cohen2022introducing}. From the very inception of reasoning foundations in medical diagnosis and \gls{cds} nearly 65 years ago---where the reasoning was based on symbolic logic and probability understanding and optimum treatment was calculated via value theory---to current advancements in AI, decision systems are being designed to model human knowledge and augment the work of clinicians \cite{ledley1959reasoning, rajkomar2019machine}.\\

\noindent This project, in collaboration with the Department of Gastroenterology and Human Nutrition at the \gls{aiims}, New Delhi, aims to develop a department-specific \gls{cdss} tailored to address the unique needs in the \gls{opd} setting, specifically for \textcolor{TUMRed}{\textbf{abdominal pain}}. The system introduces an intelligent CDSS, where a linguistic-agnostic conversational agent engages directly with patients, collecting key responses and extracting information from a protocol-driven questionnaire. This information is then presented along with a probable diagnosis and organ of origin.\\

\noindent Contemporary AI-based CDSS tends to operate as generalized, multi-purpose diagnostic tools that operate autonomously, often assuming full responsibility for diagnosis. By introducing a \textcolor{TUMRed}{\textbf{department-specific}}, \textcolor{TUMRed}{\textbf{protocol-based deterministic approach}} with AI-based linguistically agnostic conversational agents for providing probable diagnosis and organ of origin, the system ensures transparency, and accountability as the final decision remains with the physician.\\

\noindent The remainder of this report is organized as follows:
\begin{itemize}
    \item \textcolor{TUMRed}{\textbf{Section 2}} outlines the background and motivation, including abdominal pain diagnosis, clinical decision support systems, linguistics, and reasoning-based medical diagnosis.
    \item \textcolor{TUMRed}{\textbf{Section 3}} provides a detailed literature survey, identifying the state-of-the-art in CDSS and gaps.
    \item \textcolor{TUMRed}{\textbf{Section 4}} outlines the problem statement, project objectives, scope, and boundaries of the system.
    \item \textcolor{TUMRed}{\textbf{Section 5}} explains the system's design, methodology, and implementation.
    \item \textcolor{TUMRed}{\textbf{Section 6}} presents results and discussions.
    \item \textcolor{TUMRed}{\textbf{Section 7}} concludes the report with future work and recommendations.
\end{itemize}

\noindent While the text-based conversational agent has been fully developed and tested as part of this thesis, the voice-based, linguistically-agnostic functionalities are planned for implementation in the following semester


%-------------------------------------------------------------------------------
\chapter{Background and Motivation}
\label{chap:bcgmot}
\section{Background}
% \subsection{Evaluation of Abdominal Pain}
\subsection{Diagnosis of Abdominal Pain in Clinical Setting (OPD)}
\label{sec:abdominal_pain}
\lettrine{A}{ }bdominal pain is one of the most common and diagnostically challenging chief complaints encountered in clinical practice \cite{gans2015guideline}. This serves as a symptom of a wide range of underlying conditions, encompassing both gastrointestinal and non-gastrointestinal issues. This wide range of potential causes increases diagnostic complexity, requiring a systematic and comprehensive evaluation process.\\

\noindent The assessment of abdominal pain typically follows a systematic evaluation strating with a physician-patient interaction. This interaction incorporates multiple clinical dimensions. These dimensions provide clues that aid in narrowing down possible diagnoses. The key aspects considered during an evaluation include:
\begin{itemize}
    \item \textcolor{TUMRed}{\textbf{Location of Pain}}: The abdomen is anatomically divided into nine regions (as shown in Figure \ref{fig:abdominal_regions} (A) below) --- epigastric, umbilical, hypogastric, bilateral hypochondriac, bilateral lumbar, and bilateral iliac regions \cite{AbExm}. The specific region where pain is reported often serves as an essential diagnostic clue \cite{gans2015guideline}.
    \item \textcolor{TUMRed}{\textbf{Presence of Danger Signs}}: Symptoms like lightheadedness, altered sensorium, and respiratory distress may signal critical underlying conditions requiring immediate attention.
    \item \textcolor{TUMRed}{\textbf{Severity of Pain}}: Pain intensity (1-10) is subjectively reported by the patient but often correlates with the urgency of the clinical situation.
    \item \textcolor{TUMRed}{\textbf{Onset of Pain}}: The onset of pain, whether the pain arise suddenly over minutes-hours (acute) or the pain arose gradually over hours-days (insidious).
    \item \textcolor{TUMRed}{\textbf{Character of Pain}}: Pain can be classified as burning, stabbing, pin-pricking, constricting, throbbing, dull aching, or non-specific. Each pain type is linked to a distinct set of possible diagnoses.
    \item \textcolor{TUMRed}{\textbf{Duration of Pain}}: The duration of pain can be classified as either less than 3 months (acute) or more than 3 months (chronic).
    \item \textcolor{TUMRed}{\textbf{Radiation of Pain}}: The direction of pain radiation (to the back, shoulder, groin/inner thigh, arms, or neck) provides vital diagnostic insight.
    \item \textcolor{TUMRed}{\textbf{Aggravating Factors}}: Activities, such as eating, bending forward or sideways, passing stool, passing urine, menstruation, deep inspiration, or walking and exercise, can exacerbate abdominal pain.
    \item \textcolor{TUMRed}{\textbf{Associated Symptoms}}: Symptoms like fever, nausea, vomiting, constipation, diarrhea, jaundice, and changes in bowel habits serve as crucial diagnostic adjuncts. Additionally, systemic symptoms like weight loss, loss of appetite, and signs of anxiety or depression may signal specific gastrointestinal disorders.
    \item \textcolor{TUMRed}{\textbf{Comorbidities}}: Chronic illnesses such as diabetes, cardiovascular disease, kidney disease, or a history of gallstones can affect abdominal pain etiology.
    \item \textcolor{TUMRed}{\textbf{Surgical History}}: Prior surgical procedures on the gallbladder, intestines, kidneys, or uterus can influence the current presentation of abdominal pain.
    \item \textcolor{TUMRed}{\textbf{Gender-Specific Considerations}}: Conditions related to the female reproductive health, such as abnormal vaginal bleeding, menstrual irregularities, and foul discharge, play a role in the evaluation of abdominal pain.
\end{itemize}
\noindent The process of evaluating abdominal pain begins with a physician-patient interaction where history is taken to collect information on the above-mentioned aspects. This is followed by a physical examination, as shown in Figure \ref{fig:abdominal_regions} (B) below, where physicians use visual inspection and hands-on techniques such as palpation, percussion, and auscultation. The examination is often performed with the patient in a supine position with bent knees to relax the abdominal muscles and facilitate assessment \cite{AbExm, cartwright2008evaluation}.
\begin{figure}[H]
    \centering
    \includegraphics[scale=0.09]{images/abdominal_regions_exam.png}
    \caption{A. The nine regions of the abdomen. B. The step-by-step process of a patient undergoing an evaluation of abdominal pain. The dotted red box show where the linguistic-agnostic conversational application will be augmented.} 
    \label{fig:abdominal_regions}
\end{figure}

\noindent If the initial evaluation does not provide sufficient diagnostic clarity, physicians may order laboratory tests (e.g. blood work, urine analysis, etc.) or imaging studies (e.g. ultrasound, MRI, or CT scan) to gather further evidence. Based on the collective information from history, physical examination, and diagnostic tests, physicians generate a differential diagnosis --- a list of possible conditions that could explain the symptoms. This process continues until the final diagnosis is reached \cite{ddCleveland}.\\[\baselineskip]

\noindent The application developed as part of this capstone project aims to augment the physician-patient interaction phase of abdominal pain evaluation workflow. By integrating a conversational agent, the system collects essential patient information regarding the dimensions mentioned earlier to identify a probable diagnosis and organ of origin using a deterministic approach. The final decision and subsequent physical examination remain under the physician's control.

\subsection{Brief Introduction to Clinical Decision Support Systems (CDSS)}
\textcolor{TUMBlue}{\textbf{Clinical Decision Making}}\\
\noindent Effective clinical decision-making requires:
\begin{itemize}
    \item Up-to-date Medical Knowledge
    \item Access to Accurate and Complete Patient Data
    \item Good Decision-Making Skills
\end{itemize}
\noindent However, several significant changes have emerged as the healthcare landscape has evolved:
\begin{itemize}
    \item Exponential Growth of Medical Knowledge
    \item Rapid Accumulation of Patient Data
    % \item Increased Complexity in Clinical Inference
    \item Clinical Data Capture and Documentation Burden
\end{itemize}

\noindent Given the changes and the rapid advancements in health, assisting clinicians is becoming more and more important. Thus the role of Clinical Decision Support Systems (CDSS) becomes essential in modern healthcare \cite{visweswaran2022integration}. With the adoption of Artificial Intelligence (AI) in Medicine, existing physicians can handle \textcolor{TUMBlue}{\textbf{more cases}} as systems become \textcolor{TUMBlue}{\textbf{more automated}} \cite{panch2018artificial}.\\[\baselineskip]

\noindent \textcolor{TUMBlue}{\textbf{Clinical Decision Support (CDS)}}\\
\noindent Clinical Decision Support (CDS) refers to a broad set of tools, systems, and interventions aimed at enhancing clinical decision-making. By providing clinicians, healthcare workers, and patients with \textcolor{TUMRed}{\textbf{situation-specific knowledge}}, CDS supports a range of critical healthcare activities such as diagnosis, risk assessment, prognosis, and treatment selection \cite{osheroff2007roadmap}. This support can be delivered through various mediums, including reference guidelines, alerts, and evidence-based recommendations.\\[\baselineskip]

\noindent The goal of CDS is to \textcolor{TUMRed}{\textbf{bridge the gap between clinical knowledge and patient care}}. As medical knowledge expands and healthcare systems become more data-driven, CDS ensures that healthcare providers have access to timely and relevant information. This, in turn, enables more efficient and informed decision-making \cite{osheroff2007roadmap}.\\[\baselineskip]

\noindent \textcolor{TUMBlue}{\textbf{Clinical Decision Support System (CDSS)}}\\
A Clinical Decision Support System (CDSS) is a specialized type of CDS that employs \textcolor{TUMRed}{\textbf{software-based tools}} to directly aid healthcare providers in making clinical decisions. Unlike general CDS, which may include static guidelines or references, CDSS is dynamic, \textcolor{TUMRed}{\textbf{patient-centric}}, interactive and assist physicians. By integrating patient-specific information with a computerized knowledge base, CDSS generates recommendations, alerts, and assessments that assist in diagnosis, treatment, and care planning \cite{hunt1998effects}.

\begin{mdframed}[backgroundcolor=black!10]
    \centering
    \textit{“Any software designed to aid in clinical decision-making by matching patient characteristics to a computerized knowledge base to generate tailored patient-specific assessments or recommendations”}\\
    \flushright
    -  Hunt et al. \cite{hunt1998effects}
\end{mdframed}

\noindent Unlike fully autonomous AI models, CDSS systems are designed to work \textcolor{TUMRed}{\textbf{in collaboration with healthcare professionals}}, offering them interpretive insights and enabling human oversight.\\[\baselineskip]

\noindent \textcolor{TUMBlue}{\textbf{Types of Clinical Decision Support System (CDSS)}}\\
As per \cite{visweswaran2022integration} CDSS can be broadly classified into two categories as shown in Figure \ref{fig:cdss} below:
\begin{itemize}
    \item \textcolor{TUMRed}{\textbf{Knowledge-based CDSS}}: The key components of a knowledge-based AI-CDS system include a knowledge base such as expert-derived rules and an inference mechanism for clinical application such as chained inference for rules.
    \item \textcolor{TUMRed}{\textbf{Data Derived CDSS}}: The key components of a data-derived AI-CDS include a model, such as a data-derived neural network, and an inference mechanism for clinical application, such as forward propagation in a neural network model..
\end{itemize}
\begin{figure}[H]
    \centering
    \includegraphics[scale=0.12]{images/cdss.png}
    \caption{A. Illustrates the principles of decision making process followed by a physician. B. Illustrates the two types of CDSS systems, Knowledge-based and Data-derived.}
    \label{fig:cdss}
\end{figure}

\subsection{Brief Introduction to Conversational AI in Healthcare}
\lettrine{T}{ }he efforts to use natural language system for problem solving using human input can be traced back to the 1960s \cite{bobrow1964natural} and have since evolved significantly. Dialogue systems can be classified into two categories --- task oriented chatbots for specific tasks and open-domain conversational AI for general conversation. Today, conversational AI systems, both mobile and web-based, enbale human-computer interaction using natural, human-like dialogue. Unlike chatbots, which rely on scripted responses to user queries, conversational AI utilizes NLP and LLMs to enable more dynamic, context-aware, and adaptive responses \cite{jurafsky2000speech}. Chatbots follow rigid workflows, while conversational AI systems understand intent, manage multi-turn dialogues, and evolve with user interactions.
\begin{figure}[H]
    \centering
    \includegraphics[scale=0.12]{images/dialogue.png}
    \caption{Types of Dialogue Systems. Task-oriented chatbots such as ELIZA follow a rigid workflow, while conversational AI such as ChatGPT understand intent, manage multi-turn dialogues, and evolve with user interactions.}
    \label{fig:dialogue}
\end{figure}

\noindent The origin of using chatbots in healthcare can be traced back to the development of ELIZA. Joseph Weizenbaum developed one of the first medical chatbots in the 1966 at the Massachusetts Institute of Technology (MIT) \cite{haug2023artificial, weizenbaum1966eliza}. ELIZA employed pattern-matching rules to simulate therapist-like conversations with users. The system was designed to respond to user inputs by reflecting the user's statements back as questions. \\[\baselineskip]

\noindent Since then, Natural Language Processing (NLP) has progressed to the current state of Large Language Models (LLMs). These are advanced \textcolor{TUMRed}{\textbf{AI models}} specifically trained to process, understand, and generate text. This evolution has facilitated the rise of advanced conversational AI systems that can understand, generate, and respond to human language with the help of conversational agents. These agents can understand the context of a conversation, conduct multi-turn dialogue transactions, and establish a more natural interaction with users \cite{clark2019makes}. Since the release of ChatGPT, numerous LLMs---both open-source and proprietary---have been developed at unprecedented speed \cite{clusmann2023future}. These models have been applied to various domains, including healthcare.

\subsection{Brief Introduction to Linguistic-agnostic Systems in Healthcare}
\label{sec:linguistic_agnostic}
\lettrine{C}{ }omputational linguistics focuses on \textcolor{TUMRed}{\textbf{modeling human language}} using computational techniques, enabling machines to understand and process human language. The process of language understanding can be broken down into multiple hierarchical levels, each representing a critical step in the language comprehension pipeline \cite{cohen2022intelligent}. These levels, illustrated in Figure \ref{fig:linguistic_levels}, are as follows:
\begin{itemize}
    \item \textcolor{TUMRed}{\textbf{Phonology}}: This level focuses on sound patterns in language. One of the most notable linguistic tasks at this level is \textcolor{TUMRed}{\textbf{Automatic Speech Recognition (ASR)}}, also known as speech-to-text, where speech waveforms are converted into textual data. The concept of \textcolor{TUMRed}{\textbf{linguistic-agnostic}} systems relates to the ability of ASR models to understand diverse accents, dialects, and speech variations without being restricted to a particular language or pronunciation. For example, a linguistic-agnostic system can process English spoken by people from different regions (e.g., American, British, and Indian accents) with equal accuracy.
    \item \textcolor{TUMRed}{\textbf{Morphology}}: Morphology focuses on how words are formed by combining \textcolor{TUMRed}{\textbf{morphemes}}, the smallest units of meaning. For example, in the word "unhappiness," the morphemes are "un-", "happy", and "-ness".
    \item \textcolor{TUMRed}{\textbf{Syntax}}: Syntax refers to the grammatical structure of language and focuses on how words are arranged in a sentence. It identifies the \textcolor{TUMRed}{\textbf{relationships between words}} and builds a hierarchical representation of the sentence, allowing for proper interpretation
    \item \textcolor{TUMRed}{\textbf{Semantics}}: Semantics deals with \textcolor{TUMRed}{\textbf{word meanings}} and their relationships within sentences. It enables systems to understand the meaning of words, phrases, and sentences after receiving inputs from the phonology, morphology, and syntax layers
    \item \textcolor{TUMRed}{\textbf{Pragmatics}}: This layer focuses on how context influences meaning by going beyond the literal meaning of words. 
    \item \textcolor{TUMRed}{\textbf{Generation}}: While the above layers focus on language understanding, this layer focuses on \textcolor{TUMRed}{\textbf{realistic language generation}} given a computational representation 
\end{itemize}

\begin{figure}[H]
    \centering
    \includegraphics[scale=0.1]{images/linguistic_levels.png}
    \caption{Linguistic Levels}
    \label{fig:linguistic_levels}
\end{figure}

\noindent Accurate interpretation of human language---especially spoken language---is one of the most critical factors that influence the success of \textcolor{TUMRed}{\textbf{human-computer interaction}}. To achieve a \textcolor{TUMRed}{\textbf{linguistic-agnostic system}}, a large, diverse dataset is required to capture regional, phonetic, and dialectal variations in speech.\\[\baselineskip]

\noindent India's linguistic diversity presents a unique challenge for developing linguistic-agnostic systems. With more than 1,652 "mother tongues" and 22 scheduled languages, poses significant challenges for healthcare access across the health literacy spectrum \cite{languagesIndia}. Patients often speak in their native languages and dialects, and may find it challenging to describe symptoms or comprehend medical advice in a language unfamiliar to them. Addressing this challenge requires linguistic-agnostic systems capable of understanding and responding in multiple languages, dialects, and speech patterns.\\[\baselineskip]

\noindent Efforts to bridge linguistic barriers have seen notable contributions within India. The Bhashini initiative \cite{bhashini}, under India's National Language Translation Mission, aims to create linguistic-agnostic systems that allow seamless access to digital services in 22 Indian languages. Using AI-driven voice and text-based translations, Bhashini strives to reduce the language divide across India's vast population. AI4Bharat \cite{ai4bharat}, a research lab at IIT-Madras, advances AI technology for Indian languages, focusing on speech synthesis, automatic speech recognition, and natural language understanding. Their open-source models were trained on diverse Indian languages and dialects data collected from over 400 districts in India with over 15,000 hours of transcribed data, encompassing all 22 scheduled languages of India.\\[\baselineskip]

\noindent The Folk Computing project, with its visions to allow speech input and output to enhance the accessibility of health related information in multiple languages, is a step towards improving accessibility. It is an Android based application with a chat interface powered by LLMs, enabling users to access the health model in multiple languages. The inital version of the application was developed for Hindi, Tamil, Telugu, and Bengali languages at Ashoka University \cite{folkcomp}. A version with Chinese language support was also developed in collaboration with the HealthUnity organization. The demonstration of the application are available on the Folk Computing website \href{https://kutumlab.github.io/folk-comp/#demos}{here}. The complete link to the project is available in Appendix \ref{appendix:links}.

%%%%%%%%%%%%%%%%%%%%%%%%%%%%%%%%%%%%%%%%%%%%%%%%%%%%%%%%%%%%%%%%%%%%%%%%%%%%%%%%%%%%%%%%%%%%%%%%%%%%%%%%%%%%%%%%%%%%%%%%%%%%%%%%%%%%%%%%%%%%%
%%%%%%%%%%%%%%%%%%%%%%%%%%%%%%%%%%%%%%%%%%%%%%%%%%%%%%%%%%%%%%%%%%%%%%%%%%%%%%%%%%%%%%%%%%%%%%%%%%%%%%%%%%%%%%%%%%%%%%%%%%%%%%%%%%%%%%%%%%%%%
\section{Motivation}
\textcolor{TUMBlue}{\textbf{Assisting Physicians in OPD settings}}
\lettrine{T}{ }he \textcolor{TUMRed}{\textbf{Department of Gastroenterology and Human Nutrition}} at the All India Institute of Medical Sciences (AIIMS), New Delhi, was established in 1971 to create qualified gastroenterologists for the country. It is one of the 49 teaching departments and centers at AIIMS, New Delhi.\\[\baselineskip]

\noindent According to the 67th AIIMS Annual Report (2022-2023) \cite{AIIMS2024}, the department managed a total of \textcolor{TUMRed}{\textbf{1,35,944 outpatient department (OPD) cases}}, of which \textcolor{TUMRed}{\textbf{42,586 were new cases}} and \textcolor{TUMRed}{\textbf{93,358 were follow-up cases}}. This is a significant increase from the previous year's figures of 17,790 new cases and 35,622 follow-up cases as shown in Table \ref{tab:aiims_opd} below. In total, the Main Hospital at AIIMS catered to \textcolor{TUMRed}{\textbf{10,39,523 patients}} through its General OPD, Specialty Clinics, and Emergency Department.

% insert table for above data
\begin{table}[h]
    \centering
    \begin{tabular}{|c|c|c|c|}
        \hline
        \textbf{Year} & \textbf{New Cases} & \textbf{Follow-up Cases} & \textbf{Total Cases} \\
        \hline
        2020-2021 & 7,920 & 11,956 & 19,876 \\
        2021-2022 & 17,790 & 35,622 & 53,412 \\
        2022-2023 & 42,586 & 93,358 & 1,35,944 \\
        \hline
    \end{tabular}
    \caption{Number of OPD Cases (new and follow-up) observed at the Department of Gastroenterology and Human Nutrition, AIIMS, New Delhi through the years 2020-2023.}
    \label{tab:aiims_opd}
\end{table}

\noindent The routine gastroenterology OPD operates from Monday to Friday, 8:30 a.m. to 1:00 p.m. \cite{AIIMSOPD}. Given this limited time frame of 270 minutes daily over 5 working days, approximately \textcolor{TUMRed}{\textbf{8,500 new cases are handled per day}}, with multiple physicians addressing various chief complaints. As discussed in Section \ref{sec:abdominal_pain}, one of the most diagnostically challenging chief complaints is \textcolor{TUMRed}{\textbf{abdominal pain}}. A complete protocol of structured questions helps filter the probable diagnosis and the organ of origin. \textcolor{TUMRed}{\textbf{Automating}} this process will significantly assist physicians in managing the high volume of patients the OPD setting.\\[\baselineskip]

\noindent This challenge forms the motivation for the development of an autonomous system. The application aims to augment the physician-patient interaction phase of abdominal evaluation through a conversational agent and assist physicians by generating probable diagnoses and identifying the organ of origin before physical examination. When deployed, the system can handle multiple patients simultaneously via multiple devices. It can conduct the initial evaluation and generate a probable diagnosis and organ of origin. The output will be printed for the patient, who can then present it to the physician for further evaluation.\\[\baselineskip]

\noindent Another significance is the drive to automate and digitze the health data for India's linguistically diverse population. AIIMS serves as a \textcolor{TUMRed}{\textbf{melting pot for linguistically diverse populations}} from across India. This is reflected in the geographical distribution of inpatients at AIIMS during the year 2021-2022, as shown in the Figure \ref{fig:inpatient_distribution} below \cite{AIIMS2024}. This diversity spans multiple languages, dialects, accents, regional pronunciations, and dialectical variations.

\begin{figure}[H]
    \centering
    \includegraphics[scale=0.13]{images/inpatient_distribution.png}
    \caption{A pie chart showing the geographical distribution of inpatients at AIIMS, New Delhi, during the year 2021-2022. The majority of inpatients come from the states of Delhi, Uttar Pradesh, and Bihar.}
    \label{fig:inpatient_distribution}
\end{figure}

\noindent This diversity in linguistic backgrounds presents a unique opportunity to digitize \textcolor{TUMRed}{\textbf{voice-based}} data related to health from from system-patient interactions. Such digitlization may provide, in the future, valuable insights into the linguistic variations in healthcare settings in India.

%%%%%%%%%%%%%%%%%%%%%%%%%%%%%%%%%%%%%%%%%%%%%%%%%%%%%%%%%%%%%%%%%%%%%%%%%%%%%%%%%%%%%%%%%%%%%%%%%%%%%%%%%%%%%%%%%%%%%%%%%%%%%%%%%%%%%%%%%%%%%
%%%%%%%%%%%%%%%%%%%%%%%%%%%%%%%%%%%%%%%%%%%%%%%%%%%%%%%%%%%%%%%%%%%%%%%%%%%%%%%%%%%%%%%%%%%%%%%%%%%%%%%%%%%%%%%%%%%%%%%%%%%%%%%%%%%%%%%%%%%%%
\section{Problem Statement}
\lettrine{T}{ }he volume of patients at OPD settings related to gastrointestinal complaints is large. Given the time constraints (270 minutes) and the large influx of patients (around 8,500 cases per day), automation and digitlization of the initial physician-patient interaction, specifically for evaluating the chief complaint of abdominal pain, will significantly assist physicians in managing the high volume of patients in the OPD setting.

\section{Objectives}
The objectives are:
\begin{enumerate}
    \item \textcolor{TUMRed}{\textbf{Development of an application}} to assist physicians in the evaluation of abdominal pain in the OPD setting. This application will collect patient responses through a structured, protocol-driven questionnaire and generate a report with a probable diagnosis and organ of origin through a deterministic rule-based system.
    \item \textcolor{TUMRed}{\textbf{Implementing}} a conversational agent that interacts with patients to help collect responses for the questionnaire.
    \item \textcolor{TUMRed}{\textbf{Evaluation}} and assessment of the impact of the rule-based deterministic system with the conversational agent in the OPD setting for abdominal pain evaluation. The evaluation will be performed by comparing no system (baseline), option and click based system with help of a healthcare staff, and a fully conversational system.
\end{enumerate}



%-------------------------------------------------------------------------------
\chapter{Literature Review}
\label{chap:sota}
\section{Literature Survey}
\subsection{A brief history of the field}
\lettrine{T}{ }he field of diagnostic reasoning in medicine became an early focus of AI, demonstrating that AI methods could approximate human performance in tasks requiring extensive domain knowledge. Early systems were designed to model human cognition explicitly, prioritizing interpretability over mere optimization of accuracy. Such systems were particularly adept at explaining their reasoning, contrasting with modern AI systems optimized solely for prediction accuracy, often at the cost of transparency \cite{cohen2022introducing}.\\[\baselineskip]

\noindent One of the landmark systems in this domain was \textcolor{TUMRed}{\textbf{MYCIN}}, introduced in the 1970s \cite{cohen2022intelligent, shortliffe2012computer, shortliffe1975model}. MYCIN was developed to assist physicians in selecting appropriate antimicrobial therapies for severe infections. Its key components included:
\begin{itemize}
    \item \textcolor{TUMRed}{\textbf{Consultation Program}}: Acquired patient data and provided treatment recommendations.
    \item \textcolor{TUMRed}{\textbf{Explanation Program}}: Generated English-language explanations, detailing why certain questions were asked and how conclusions were reached.
\end{itemize}

\noindent MYCIN's knowledge of infectious diseases was represented as production rules—conditional statements linking observations to inferred outcomes. It introduced backward chaining, a reasoning strategy that began with a hypothesis and worked backward to validate it using available evidence. This allowed MYCIN to answer "WHY" questions, making its decision-making process interpretable and user-friendly \cite{shortliffe1975model, musen2021clinical}.\\[\baselineskip]

\noindent Before MYCIN, the Leeds Abdominal Pain System, developed in the late 1960s, marked another early attempt at diagnostic reasoning. F. T. de Dombal and his colleagues at the University of Leeds created decision aids for diagnosing abdominal pain based on Bayesian probability theory. This work laid the foundation for probabilistic reasoning in medical decision-making \cite{musen2021clinical}.\\[\baselineskip]

\noindent The initial wave of AI-driven diagnostic systems emphasized the trade-off between accuracy and interpretability, often prioritizing the latter. Over time, the focus shifted to integrating AI into a larger system involving human decision-makers, aiming to improve the quality, efficiency, and safety of clinical practice.

\subsection{The Era of Knowledge-Based Systems (KBS)}
\lettrine{F}{ }ollowing systems like MYCIN and the Leeds Abdominal Pain System, the next phase of AI in healthcare saw the rise of Knowledge-Based Systems (KBS). These systems sought to replicate human reasoning in complex medical scenarios by formalizing knowledge into computational representations. The core ideas of KBS were:
\begin{enumerate}
    \item \textcolor{TUMRed}{\textbf{Representing knowledge using}}:
    \begin{itemize}
        \item Formal methods: Mathematical frameworks for precise reasoning.
        \item Ontological commitments: Hierarchies of concepts organized logically.
        \item Fragmentary theories of reasoning: Integrating logic, psychology, biology, statistics, and economics.
    \end{itemize}
    \item \textcolor{TUMRed}{\textbf{Ensuring}} that representations were computationally efficient and intuitive for human practitioners to understand and modify.
\end{enumerate}
\noindent Prominent approaches to knowledge representation included Rules and Patterns—Logical or heuristic rules for decision-making. Probabilistic Models—Methods like Naive Bayes, Bayesian Networks, and Influence Diagrams. Causal Mechanisms—Explaining outcomes through cause-and-effect relationships. Fuzzy Logic—Handling uncertainty and imprecision in medical reasoning.\\[\baselineskip]
\noindent The success of KBS depended on robust methods for acquiring and organizing medical knowledge. Techniques included:
\begin{itemize}
    \item \textcolor{TUMRed}{\textbf{Taxonomic Ontologies}}: Organizing concepts into hierarchical structures, specifying their super- and sub-categories.
    \item \textcolor{TUMRed}{\textbf{Knowledge Graphs}}: Capturing relationships between concepts for intuitive reasoning.
    \item \textcolor{TUMRed}{\textbf{Textual Co-occurrence Analysis}}: Identifying relationships from sentences, paragraphs, or articles.
    \item \textcolor{TUMRed}{\textbf{Unified Medical Language Systems (UMLS)}}: Leveraging the Metathesaurus to bridge semantic gaps without constructing exhaustive ontologies.
    \item \textcolor{TUMRed}{\textbf{Triple-Based Models}}: Representing knowledge as ([concept] - [relation] - [concept]) triples, enabling systems to address "which," "why," and "does" questions \cite{demner2009can}.
\end{itemize}

\noindent Some other methods included conception and prototypical design of a decision-support server \cite{eich1999internet}. These methods were observed to have a significant impact on user-friendliness and performance of the system.
\section{State of the Art}
\lettrine{T}{ }he evolution of AI technologies in healthcare has driven a shift toward more dialogue-based approaches, taking advantage of intelligent agents to facilitate interactive conversations. are designed to interact with humans using dialog systems, which are computational frameworks for managing dialog. The development of such technologies is particularly critical in healthcare, where the delivery of complex information must be accessible. These systems can play a transformative role for individuals with low health literacy or limited familiarity with technology, helping them navigate complex medical advice and treatment options effectively \cite{cohen2022intelligent}.\\[\baselineskip]

\noindent Recent advancements in conversational LLMs like GPT-4 have demonstrated remarkable capabilities in reasoning, planning, and using contextual information. While GPT-4 is trained on vast amounts of text data, and primarly developed for general domain, it lacks the specialized medical knowledge required for clinical decision-making. Clinical data tends to be proprietary, sensitive, and subject to strict privacy regulations, making it challenging to access and use for training AI models. Because of the interactive nature of the system, the user can request more detail regarding the response by asking follow-up questions or asking for more concise responses in order to get “to the point” more rapidly \cite{lee2023benefits}.\\[\baselineskip]


\noindent On the other hand, systems like Google AMIE: A research AI system for diagnostic medical reasoning and conversations \cite{karthikesalingam2024amie} represent a new generation of AI tailored specifically for medical diagnostics and reasoning. Google AMIE is a research-focused AI designed to:
\begin{itemize}
    \item \textcolor{TUMRed}{\textbf{Perform Diagnostic Medical Reasoning}}: Simulating primary care physician (PCP) consultations and Objective Structured Clinical Examinations (OSCE).
    \item \textcolor{TUMRed}{\textbf{Modular Architecture}}: AMIE utilizes a stage and agents-based architecture, allowing for easy integration of new modules and agents.
\end{itemize}
\noindent A notable emerging trend also being observed where graphs-based models and graph representation learning \cite{johnson2024graph} is being used in medicine to manage and analyze the vast, multimodal datasets generated in healthcare. Graph-based models are particularly well-suited for capturing complex relationships between entities, such as diseases, symptoms, treatments, and patient demographics. 

\section{Gap Analysis}

\lettrine{A}{ } significant limitation of current approaches is their black-box nature and full autonomy in diagnostic decisions \cite{clusmann2023future}. There is a pressing need for auditable and traceable systems where AI assists rather than replaces clinical decision-making \cite{cohen2022intelligent}. The ideal solution should provide explainable recommendations while keeping diagnostic control firmly in the hands of healthcare professionals, ensuring accountability, and maintaining the critical role of human expertise in patient care as explained by the three pillars in Figure \ref{fig:explainable_AIM} \cite{explainableAIM}. This is particularly important in specialized medical departments where standardized protocols and human oversight are essential for patient safety \cite{cohen2022intelligent}.\\[\baselineskip]

\noindent While recent advancements in LLMs have shown impressive capabilities in medical domains, there remains a critical gap in \textcolor{TUMRed}{\textbf{department-specific}} Clinical Decision Support Systems. Current general-purpose AI systems lack the specialized knowledge and protocols required for specific medical departments, potentially leading to misinformation and compromised patient care. Additionally, most existing systems operate as standalone solutions rather than taking a system-level perspective that considers integration with existing clinical workflows and protocols, incorporating both \textcolor{TUMRed}{\textbf{rule-based}} and \textcolor{TUMRed}{\textbf{data-driven}} approaches while also ensuring \textcolor{TUMRed}{\textbf{explainability}} and \textcolor{TUMRed}{\textbf{transparency}} in decision-making.

\begin{figure}[h]
    \centering
    \includegraphics[scale=0.1]{images/explainable_AIM.png}
    \caption{Three pillars of explainable AI}
    \label{fig:explainable_AIM}
\end{figure}


%-------------------------------------------------------------------------------
\chapter{Content}
\label{chap:content}
\section{Problem Statement}
\lettrine{T}{ }he Department of Gastroenterology and Human Nutrition at AIIMS, New Delhi, handles a high influx of outpatient cases daily. Due to time constraints and the high volume of patients, physicians have limited capacity to conduct comprehensive, protocol-driven evaluations for each patient. For the chief complaint of abdominal pain, a structured diagnostic approach is required. This process typically begins with identifying the pain rating and location, followed by questions on aggravating factors, associated symptoms, and the patient’s medical history. However, given the volume of patients, it becomes difficult for physicians to conduct a complete assessment for every individual.\\[\baselineskip]

\noindent To ensure patient care quality, there is a need for a support system that can streamline the diagnostic process by guiding the patients through initial screening and decreasing the cognitive burden on healthcare providers.\\[\baselineskip]

\noindent The secondary problem is the need for linguistically agnostic systems. Since patients at AIIMS come from diverse linguistic backgrounds, it is essential to create a system that can interact with patients regardless of their language proficiency.

\section{Objectives}
The primary objectives of this project are as follows:
\begin{itemize}
    \item \textcolor{TUMRed}{\textbf{Development of a Department-Specific CDSS}}: To design and develop a \textcolor{TUMBlue}{department-specific}, intelligent \textcolor{TUMBlue}{clinical decision support system} for the screening of \textcolor{TUMBlue}{abdominal pain}.  This system will identify the \textcolor{TUMBlue}{probable diagnosis} and organ of origin based on a structured, protocol-driven questionnaire to aid physicians.
    \item \textcolor{TUMRed}{\textbf{Design of a Linguistic-Agnostic Conversational Agent}}: To design and develop a conversational agent that is linguistic-agnostic to drive above mentioned CDSS.
    \item \textcolor{TUMRed}{\textbf{Deployment and Integration}}: To deploy and integrate the system in the OPD setting at the Department of Gastroenterology and Human Nutrition, AIIMS, New Delhi.
\end{itemize}
The project also has a set of secondary objectives, which are as follows:
\begin{itemize}
    \item \textcolor{TUMRed}{\textbf{Collection of Linguistic Data}}: To collect linguistic data from patients at AIIMS, representing diverse accents, dialects, and language variations. This data will be used to improve the robustness of linguistic-agnostic models for future use.
    \item \textcolor{TUMRed}{\textbf{Evaluation of a Rule-Based and Generative Hybrid System}}: To develop and evaluate a hybrid approach that combines a rule-based, deterministic decision-making engine with a generative AI conversational agent for more robust and explainable clinical support.
\end{itemize}
\section{Scope \& Boundaries}
\subsection{Scope}
\lettrine{T}{ }his project focuses on the development of a department-specific Clinical Decision Support System (CDSS) for the Department of Gastroenterology and Human Nutrition at AIIMS, New Delhi. The primary goal of the system is to support the initial evaluation of patients presenting with abdominal pain.\\[\baselineskip]

The system does not autonomously diagnose patients. Instead, it provides probable diagnoses and suggests a possible organ of origin. Final diagnostic authority remains with the physician, who can audit and modify the system's suggestions based on clinical judgment. It has Autonomous Decision-Making and follows a human-in-the-loop approach.

\subsection{Boundaries}
\lettrine{T}{ }he system is limited to the initial evaluation and probable diagnosis of abdominal pain and does not include physical examination components. It is tailored to the Department of Gastroenterology and Human Nutrition's protocol for abdominal pain at AIIMS, New Delhi, and does not generalize to other medical departments or conditions. Additionally, the system does not make final diagnostic decisions, as the ultimate responsibility for diagnosis rests with the physician.

%--------------------------------------------------------------------------------
\chapter{Methodology}
\label{chap:methodology}
\section{Protocol}
\lettrine{T}{ }he development of the Clinical Decision Support System (CDSS) was guided by a structured protocol provided by the physicians of the Department of Gastroenterology and Human Nutrition at AIIMS, New Delhi. This protocol defines the flow of patient interaction through a set of questions designed to capture essential diagnostic information related to abdominal pain.\\[\baselineskip]

\noindent The initial protocol consisted of a 12-question diagnostic questionnaire that addressed key clinical dimensions of abdominal pain, as described in Section \ref{sec:abdominal_pain}. These questions explored aspects like pain location, intensity, aggravating factors, associated symptoms, etc. However, for a more controlled and streamlined workflow, the questionnaire was later broken down to include 17 refined questions. By refining these questions, a categorization of questions was made possible. For instance, questions related to menstrual cycle health were separated from the general set of symptoms to provide more focused and relevant insights for female patients.\\[\baselineskip]

\noindent Through refinement, the 17 questions were grouped into the following categories:
\begin{enumerate}
    \item \textcolor{TUMRed}{\textbf{Discriminators}}: Critical questions are used to differentiate between emergency cases and regular OPD cases. If the answer to any of these questions indicates a potential emergency, the patient is automatically redirected to the emergency department instead of continuing through the normal OPD process.
    \item \textcolor{TUMRed}{\textbf{Demographic}}: Questions related to the patient's demographic information, such as age and gender.
    \item \textcolor{TUMRed}{\textbf{Female-Specific}}: Questions specific to female health issues --- menstrual health --- that play a significant role in the diagnostic process for women but are irrelevant to male patients.
    \item \textcolor{TUMRed}{\textbf{General}}: These are common questions applicable to all patients, regardless of gender, and form the core of the diagnostic questionnaire.
\end{enumerate}
A detailed breakdown of the 17 questions into the four categories is presented in Figure \ref{fig:question_breakdown}. A tabulated version of the 17 questions and their respective answers can be found in Appendix \ref{appendix:questionnaire}.

\begin{figure}[h]
    \centering
    \includegraphics[scale=0.1]{images/question_breakdown.png}
    \caption{Breakdown of the 17 Questions into Categories}
    \label{fig:question_breakdown}
\end{figure}

\section{Probable Diagnosis and Organ of Origin}
\lettrine{T}{ }he core function of the CDSS is to map patient responses to probable diagnoses and the organ of origin. The process for determining these outcomes relies on a pre-defined list of possible diagnoses and corresponding organs of origin, as provided by the Department of Gastroenterology and Human Nutrition at AIIMS.

\subsection{Diagnosis Mapping}
The Physician provided a list of 29 possible diagnoses related to abdominal pain. Each diagnosis is associated with a specific set of responses to the 17 questions from the protocol. To identify a potential diagnosis, the patient's responses are compared against this set of known answer combinations. If the patient's responses align with one of these pre-defined answer sets, a match is established for the probable diagnosis.

\subsection{Organ-of-Origin Identification}
The CDSS performs the identification of the organ of origin for abdominal pain. This identification is done via the mapping established by linking 29 diagnoses to a set of 19 possible organs of origin. Once a probable diagnosis is identified, the system determines the corresponding organ of origin using a \textcolor{TUMBlue}{deterministic mapping} approach. This allows the system to be interpretable.\\[\baselineskip]

\noindent The relationships between questions, diagnoses, and organs of origin are represented as a graph structure. Each diagnosis is connected to potential answers for the 12 core questions, forming distinct paths in the graph. The graphical representation in Figure \ref{fig:graphical_representation} provides a visual explanation of how diagnoses are reached based on patient responses. A tabulated mapping of organs of origin to diagnoses can be found in Appendix \ref{appendix:organs_of_origin}.\\[\baselineskip]

\noindent In Figure \ref{fig:graphical_representation}, the pink nodes (center) correspond to the 29 diagnoses, while the yellow nodes represent the 19 possible organs of origin (center below). The nodes representing answers to the 12 original questions are displayed in 12 distinct colors, each corresponding to a specific question.

\begin{figure}[H]
    \centering
    \includegraphics[scale=0.07]{images/graphical_representation.png}
    \caption{Graphical Representation of the Mapping between Questions, Diagnoses, and Organs of Origin}
    \label{fig:graphical_representation}
\end{figure}

\subsection{Workflow}
The workflow of the questionnaire begins with the patient's interaction with the system through a structured questionnaire. As shown in Figure \ref{fig:workflow}, it starts with discriminator questions to identify emergencies; if flagged, the patient is directed to the emergency department. Otherwise, the patient proceeds to demographic questions to gather basic information like age and gender, followed by general questions covering universal symptoms and history. Female patients are asked additional questions related to menstrual health before proceeding to the general questions. The patient's responses are then mapped to probable diagnoses and organs of origin, as described in the previous section. The system generates a report based on the patient's responses, which contains all the answers provided by the patient, the probable diagnosis, and the organ of origin for the physician's reference.
\begin{figure}[h]
    \centering
    \includegraphics[scale=0.1]{images/workflow.png}
    \caption{Workflow of the Patient-Questionnaire Interaction}
    \label{fig:workflow}
\end{figure}

\section{Implementation of the CDSS}
\lettrine{T}{ }he CDSS was implemented using a comprehensive technology stack. The aim was to create a system that is user-friendly, efficient, and easily accessible to physicians. The system was designed to be integrated into the hospital's existing infrastructure, allowing for seamless interaction with the CDSS.\\[\baselineskip]

\noindent The implementation involved the creation of both an Android application and a web application for patient interaction. The initial version Android application was developed using Java and Kotlin in Android Studio, while the web application was built using Streamlit in Python. Streamlit is an open-source Python framework for data scientists and AI/ML engineers to deliver dynamic data apps \cite{streamlit}. 

\subsection{Frontend Development and User Interface}
The frontend development involved creating a user-friendly interface for patient interaction. The Android application was designed to be intuitive and easy to navigate, with a clean and simple layout. The web application was developed to provide a seamless experience for patients interacting with the system on a desktop or mobile device. The frontend was designed to be responsive, ensuring that the system is accessible across a wide range of devices.\\[\baselineskip]

\noindent The initial prototype of both the Android application was designed using Figma, a web-based design tool. Both applications were designed to follow the workflow defined in Figure \ref{fig:workflow}. Both applications followed a click-through option for patient interaction, where the patient could navigate through the questionnaire by answering each question sequentially. The responses were stored in the system and used to generate the probable diagnosis and organ of origin. As seen in \cite{eich1999internet}, even a simple interface with a well-defined workflow can be effective and have a significant impact on user-friendliness.\\[\baselineskip]

\noindent The demonstration of the Android application design can be found on our website \href{https://kutumlab.github.io/abdominal-pain-cdss/#mobile-application}{here}. Another demonstration of the web application for a successful probable diagnosis and organ of origin can be found \href{https://kutumlab.github.io/abdominal-pain-cdss/#web-application-diagnosis}{here}. The web application for an emergency department visit can be found \href{https://kutumlab.github.io/abdominal-pain-cdss/#web-application-emergency-visit}{here}. The screenshots and complete URLs of the same are shown in the Appendix \ref{appendix:frontend_screenshots}. A future version of the CDSS will include the interface for voice-based interaction

\subsection{Backend Development}
The backend of the CDSS has two major components: the data dictionary and the conversational agent. The two components work together via a Python script. The entire backend is containerized using Docker for easy deployment and management.

\subsubsection{Data Dictionary}
Multiple data dictionaries contain the mapping between patient responses and probable diagnoses and organs of origin. These dictionaries are used to identify the probable diagnosis and organ of origin based on the patient's responses. Dictionaries are stored in JSON format for easy access and retrieval.
\subsubsection{Conversational Agent}
The conversational agent is responsible for interacting with the patient and guiding them through the questionnaire. The agent asks questions based on the patient's responses and stores the answers for further processing. The design of the conversational agent is shown in Figure \ref{fig:conversational_agent}.

\begin{figure}[H]
    \centering
    \includegraphics[scale=0.15]{images/conversational_agent.png}
    \caption{Creation of Conversational Agent}
    \label{fig:conversational_agent}
\end{figure}

\noindent We create a conversational agent using the open source Large Language Models (osLLMs) like Mistral (\texttt{Mistral 7B})\cite{jiang2023mistral}, Llama (\texttt{Llama 3.1 8B}) \cite{dubey2024llama}, Gemma (\texttt{Gemma 2 9B}) \cite{team2024gemma}, and others. The models were selected based on their performance and parameter size. From the above-mentioned models, \texttt{Gemma 2 9B} was selected for the conversational agent based on its performance and parameter size. We use prompt engineering \cite{liu2023pre} to provide the conversational agent with the necessary context to extract the patient's responses as accurately as possible after asking each question with its respective answer choices. \\[\baselineskip]

\noindent The conversational agent is hosted and managed by using Ollama \cite{ollama}, an open-source platform that allows for the deployment and management of large language models. Ollama provides an efficient environment for running these models locally, ensuring privacy and faster processing. Ollama provides an efficient environment for running these models locally, ensuring privacy and faster processing. The model is queried with the patient's responses using a REST API. We used Nvidia RTX A5000 GPU for loading the model and processing the patient's responses.

\subsection{System Design}
The system design of the CDSS is shown in Figure \ref{fig:high_level_design}. The system consists of two main components: the frontend and the backend. The frontend includes the Android and web applications, while the backend includes the data dictionary and the conversational agent. The containerized backend can be deployed on a local server or cloud platform such as AWS or Google Cloud.
\begin{figure}[H]
    \centering
    \includegraphics[scale=0.19]{images/high_level_design.png}
    \caption{High-Level Design of the CDSS}
    \label{fig:high_level_design}
\end{figure}

\subsection{Augmented Workflow}
The high-level augmentation of CDSS into the existing workflow of evaluation of abdominal pain is shown in Figure \ref{fig:abdominal_regions} (B), depicted by the dotted box. A more detailed view of the augmented workflow is shown in the Figure \ref{fig:aug_workflow}.
\begin{figure}[H]
    \centering
    \includegraphics[scale=0.12]{images/aug_workflow.png}
    \caption{CDSS Augmented Workflow for Abdominal Pain Evaluation}
    \label{fig:aug_workflow}
\end{figure}

\noindent The patient interacts with the CDSS through the Android or web application, answering a set of structured questions. The responses are processed by the conversational agent, which identifies the probable diagnosis and organ of origin based on the patient's responses. The system generates a report containing the patient's responses, probable diagnosis, and organ of origin. The report is then reviewed by the physician, who uses the information to further the diagnostic process. The auditing of responses and the probable diagnosis is done by the physician to ensure the accuracy of the system. In case of any discrepancies, the physician can ask the patient for additional information to refine the diagnosis.

\subsection{Challenges \& Mitigations}
One of the Challenges of any AI-based CDSS is the issue of explainability. The compartmentalization of the system ensures that the components can be updated independently without affecting the overall functionality of the system. It also ensures that the system can be easily audited and the output of each component can be tracked providing transparency, explainability, and interpretability. In case of any conflicting diagnosis, the output of the system can be easily traced back to the patient's responses and the mapping between the responses and the probable diagnosis.\\[\baselineskip]

\noindent Another challenge is the issue of privacy and security. The system is designed to be deployed on a local server or cloud platform, ensuring that patient data is stored securely and complies with all relevant data protection regulations. The open-source nature of the system allows for easy customization and mitigates the issue of privacy and security. The system can be easily adapted to meet the specific requirements of different departments or hospitals.\\[\baselineskip]

\section{Evaluation}
\lettrine{C}{ }urrently, the CDSS is in the final stages of development and is yet to be deployed for clinical evaluation. The evaluation process will involve a beta test with a small group of physicians from the Department of Gastroenterology and Human Nutrition at AIIMS. The physicians will interact with the system and provide feedback on the system's usability, accuracy, and overall performance. The feedback will be used to refine the system further before a full-scale deployment.\\[\baselineskip]

%-------------------------------------------------------------------------------
\chapter{Results \& Discussions}
\label{chap:results}
% Results and Discussion
\section{Results}
% The initial version of Clinical Decision Support System (CDSS) implementation successfully integrated a protocol-driven questionnaire with a linguistic-agnostic conversational agent. The system aims to map patient responses from a 17-question diagnostic questions (categorized into 4 groups) to probable diagnoses and corresponding organs of origin using a deterministic approach for traceable decision-making. The system's compartmentalized architecture --- Android and web-based frontends alongside a containerized backend powered by Google's Gemma 2 9B model, showcases effective modularity and interpretability. 
\lettrine{T}{ }he Clinical Decision Support System (CDSS) implementation successfully addresses the challenge of conducting comprehensive protocol-driven evaluations in high-volume OPD settings. The system integrates a 17-question diagnostic protocol (categorized into discriminators, demographic, female-specific, and general questions) with a linguistic-agnostic conversational agent powered by Google's Gemma 2 9B model. The system's compartmentalized architecture features both Android and web-based frontends alongside a containerized backend, enabling deterministic mapping between patient responses and probable diagnoses with their corresponding organs of origin. This approach ensures transparency and traceability in the decision-making process while maintaining system modularity and interpretability.


\section{Discussion}
% The department-specific approach of the CDSS, as opposed to a general-purpose diagnostic tool, shows promise in addressing the unique needs of AIIMS' Department of Gastroenterology and Human Nutrition. The hybrid approach, combining rule-based decision making with AI-powered conversation handling, offers advantages over purely AI-driven systems by following a human-in-the-loop approach. The system will be evaluated in a clinical setting to assess its impact on patient flow. The successful implementation of text-based interaction lays the groundwork for future voice-based linguistic-agnostic functionality.
\lettrine{T}{ }he department-specific CDSS demonstrates potential in streamlining the systematic evaluation of abdominal pain, which typically requires assessment across multiple clinical dimensions including pain location, intensity, character, duration, and associated symptoms. By automating the initial information-gathering phase, the system supports physicians in following the structured diagnostic framework while managing high patient volumes. The hybrid approach, combining rule-based decision-making with AI-powered conversation handling, maintains human oversight in the diagnostic process - a crucial factor in clinical settings. Unlike autonomous AI systems, this human-in-the-loop approach ensures that final diagnostic decisions remain with physicians while providing them with organized patient responses and probable diagnoses. While the current text-based implementation shows promise, clinical evaluation at AIIMS will assess its impact on patient flow and diagnostic efficiency. The system's foundation also supports future expansion into voice-based linguistic-agnostic functionality, potentially improving accessibility for India's linguistically diverse patient population.


%-------------------------------------------------------------------------------
\chapter{Conclusion \& Future Work}
\label{chap:conclusion}
% Conclusion and Future Work
\section{Conclusion}
\lettrine{T}{ }he evaluation of abdominal pain remains one of the most diagnostically challenging tasks in clinical practice, requiring systematic assessment across multiple dimensions. In high-volume settings like the Department of Gastroenterology and Human Nutrition at AIIMS, New Delhi, where physicians handle over 1,35,000 OPD cases in 2022-23 alone, conducting comprehensive protocol-driven evaluations for each patient becomes increasingly difficult. While AI-driven Clinical Decision Support Systems offer potential solutions, their black-box nature and autonomous decision-making raise concerns about explainability and accountability in healthcare settings.\\[\baselineskip]

\noindent This project presents an approach that combines the benefits of structured diagnostic protocols with modern AI capabilities. Through collaboration with AIIMS, the project aims to develop a department-specific CDSS that uses a deterministic mapping approach to link patient responses from a 17-question diagnostic protocol to probable diagnoses and organs of origin. The system's compartmentalized architecture, featuring both Android and web-based interfaces backed by a containerized backend with open source large language models (osLLMs) powered conversational agent, ensures transparency and traceability in the decision-making process.\\[\baselineskip]

\noindent The system shows how AI can augment the clinical decision-making by maintaining human oversight while streamlining initial patient evaluation. By following a human-in-the-loop approach, the CDSS supports physicians in managing high patient volumes while ensuring that final diagnostic authority remains with healthcare professionals. 
% The successful implementation of the option-based input prototype system, designed to handle diverse patient populations at AIIMS, establishes a foundation for more accessible and efficient clinical support tools.

\section{Future Work}
\lettrine{F}{ }uture development of the system will focus on several key areas. First, the osLLM Agent will be added to the current prototypes of option-based input system to provide a conversational interface for patients. This will involve integrating the osLLM Agent with the existing system to enable natural language conversations with patients. Additionally,the Android and web applications will be enhanced with additional features including voice-based input, and final reports generation.\\[\baselineskip]

\noindent Future future direction is the design and development of the linguistic-agnostic conversational models. This will involve design and development of models leveraging the diverse patient population at AIIMS to create robust models capable of handling various Indian languages, accents, and dialects. \\[\baselineskip]


\noindent Current plan is to encorporate the support of openCHA framework \cite{abbasian2023conversational} for empowering the conversational agent with voice-based functionality and external knowledge support with built-in translation tools as a part of the system. This starting point will be used to develop a more comprehensive linguistic-agnostic conversational agent capable of handling diverse patient populations.\\[\baselineskip]

% \noindent Another future direction is providing probable diagnoses and organ-of-origin of the symptoms that do not map directly to the current data dictionary. This will involve expanding the system's knowledge base using existing medical terminologies, vocabularies, and ontologies. This will include, for example, mapping the abdominal pain codes from ICD-10 to the symptoms from disease ontology and desiginig a mechanism to provide probable diagnoses for such symptoms.\\[\baselineskip]

\noindent The final phase will involve comprehensive deployment, integration, and evaluation at AIIMS. This will include rigorous testing of the system's impact on patient flow, and diagnostic efficiency. The evaluation will also assess the impact of the system on the clinical workflow for abdominal pain evaluation.


%-------------------------------------------------------------------------------
% Bibliography
%-------------------------------------------------------------------------------
\backmatter
% unsrturl-custom works only with hyperref loaded
% use other styles like unsrt when hyperref is not loaded
\bibliographystyle{unsrturl-custom}
\bibliography{src/bibliography}


%-------------------------------------------------------------------------------
% Appendix
%-------------------------------------------------------------------------------
\appendix
\chapter{Appendix}
\section{Probable Diagnosis and Organ of Origin}
\label{appendix:organs_of_origin}
Given below is a table mapping the 19 organs of origin to the 29 probable diagnoses. The Table \ref{tab:organs_of_origin} is structured such that each row corresponds to a organ of origin, and each value in column \texttt{Probable Diagnoses} corresponds to a (list of) probable diagnosis(es) that can originate from that organ. 
\begin{table}[htbp]
    \centering
    \resizebox{\textwidth}{!}{ % Resize entire table
        \begin{tabular}{|p{1cm}|p{3cm}|p{12cm}|} 
            \hline
            \textbf{No.} & \textbf{Organ of Origin}          & \textbf{Probable Diagnoses}                                                                                                 \\ \hline
            1.               & Gall Bladder, Biliary Ducts       & Biliary Pain                                                                                                                \\ \hline
            2.               & Liver \& Gall Bladder             & Rule Out Hepatobiliary Malignancy                                                                                           \\ \hline
            3.               & Biliary System                    & Acute Cholangitis                                                                                                           \\ \hline
            4.               & Liver                             & \begin{tabular}[c]{@{}l@{}}1. Liver Abscess\\ 2. Acute Hepatitis\end{tabular}                                               \\ \hline
            5.               & Pancreas                          & \begin{tabular}[c]{@{}l@{}}1. Acute Pancreatitis\\ 2. Chronic Pancreatitis\\ 3. Rule Out Pancreatic Malignancy\end{tabular} \\ \hline
            6.               & Stomach                           & Gastritis, Dyspepsia                                                                                                        \\ \hline
            7.               & Heart                             & Cardiac Pain                                                                                                                \\ \hline
            8.               & Spleen                            & Spleen Related                                                                                                              \\ \hline
            9.               & Kidney                            & \begin{tabular}[c]{@{}l@{}}1. Renal/Ureteric Calculi\\ 2. Pyelonephritis\end{tabular}                                       \\ \hline
            10.              & Appendix                          & Appendicitis                                                                                                                \\ \hline
            11.              & Intestine                         & \begin{tabular}[c]{@{}l@{}}1. Diverticulitis\\ 2. Intestinal Obstruction\\ 3. Mesenteric Ischemia\end{tabular}              \\ \hline
            12.              & Disorder of Gut Brain Interaction & Disorder of Gut Brain Interaction                                                                                           \\ \hline
            13.              & Uterus \& Ovary                   & Dysmenorrhoea                                                                                                               \\ \hline
            14.              & Uterus                            & \begin{tabular}[c]{@{}l@{}}1. Pelvic Inflammatory Disease\\ 2. Fibroid\\ 3. Adenomyosis\end{tabular}                        \\ \hline
            15.              & Adnexa                            & \begin{tabular}[c]{@{}l@{}}1. Ectopic Pregnancy\\ 2. Adnexal Lesion\end{tabular}                                            \\ \hline
            16.              & Urinary bladder                   & \begin{tabular}[c]{@{}l@{}}1. Cystitis\\ 2. Bladder stones\end{tabular}                                                     \\ \hline
            17.              & Metabolic                         & Look for Metabolic Causes: Diabetes, Hyperparathyroidism, Lead Intoxication, Porphyria                                      \\ \hline
            18.              & Abdominal Wall                    & Neuropathic Pain (Herpes, etc), Abdominal Wall Hematoma                                                                     \\ \hline
            19.              & Pleural Disorders                  & Pleural Effusion/Pleurodynia                                                                                                \\ \hline
        \end{tabular}
    }
    \caption{Mapping of Organs of Origin to Probable Diagnoses.}
    \label{tab:organs_of_origin}
\end{table}

%%%%%%%%%%%%%%%%%%%%%%%%%%%%%%%%%%%%%%%%%%%%%%%%%%%%%%%%%%%%%%%%%%
\section{Questionnaire}
\label{appendix:questionnaire}
The 17 questions used in the diagnostic questionnaire are listed below along with the possible answers. The Table \ref{tab:questionnaire} is structured such that each row corresponds to a question, and each value in column \texttt{Possible Answers} corresponds to a (list of) possible answer(s) to that question.
\begin{table}[htbp]
    \centering
    \resizebox{\textwidth}{!}{ % Resize entire table
        \begin{tabular}{|p{1cm}|p{5cm}|p{8cm}|}
            \hline
            No. & Question                                                                                                                               & Possible Answers (seperated by semi-colon)                                                                                                                                                                                                                                                                                              \\ \hline
            1       & Please rate the severity of the pain on a scale of 1 to 10                                                                             & 1-3; 4-7; 8-10                                                                                                                                                                                                                                                                                                                          \\ \hline
            2       & Have you experienced any Trauma?                                                                                                       & Yes; No                                                                                                                                                                                                                                                                                                                                 \\ \hline
            3       & Are there any danger signs present?                                                                                                    & Light Headedness; Altered Sensorium; Respiratory Distress; None                                                                                                                                                                                                                                                                         \\ \hline
            4       & Choose your gender                                                                                                                     & Male; Female                                                                                                                                                                                                                                                                                                                            \\ \hline
            5       & Choose your age group                                                                                                                  & 0-15; 15-25; 25-45; 45-60; 60+                                                                                                                                                                                                                                                                                                          \\ \hline
            6       & Where does the pain occur in the abdomen?                                                                                              & \textless{}Image of 9 regions of Abdomen\textgreater{}                                                                                                                                                                                                                                                                                  \\ \hline
            7       & How did the pain start?                                                                                                                & Over Minutes to Hours (Acute); Over Hours to Days (Insidious)                                                                                                                                                                                                                                                                           \\ \hline
            8       & What is the character of your pain?                                                                                                    & Burning Pain; Stabbing Pain; Pin Pricking Pain; Constricting Pain; Throbbing Pain; Dull aching/non-specific Pain                                                                                                                                                                                                                        \\ \hline
            9       & What's the pattern of your pain?                                                                                                       & \textless{}Image of 4 patterns of pain\textgreater{}                                                                                                                                                                                                                                                                                    \\ \hline
            10      & How long have you been experiencing the pain?                                                                                          & Less than 3 months; More than 3 months                                                                                                                                                                                                                                                                                                  \\ \hline
            11      & Does the pain radiate to any other areas?                                                                                              & No Radiation; To Back; To Shoulder; To Groin/Inner Thigh; To Arms/Neck                                                                                                                                                                                                                                                                  \\ \hline
            12      & Does the pain increase or decrease with the following aggravating/relieving factors?                                                   & No Aggravating/Relieving Factors; With Food Intake; Bending Forward; Passing Stool; Passing Urine; Menstruation; Bending Sideways; Deep Inspiration; Walking and Exercise                                                                                                                                                               \\ \hline
            13      & Are there any associated symptoms with your pain? Please specify if you experience none, or if you have any of the following symptoms. & None; Lump; Fever or Chills; Nausea and/or Vomiting; Abdominal Bloating; Constipation; Diarrhoea; Blood in Stools/Black Stools; Jaundice; Burning Micturition; Blood in Urine; Weight Loss; Loss of Appetite; Stress, Anxiety, Depression, Palpitation; Shortness of Breath; Swelling in Neck, Axilla; Skin Changes Over Abdominal Wall \\ \hline
            14      & Do you have any comorbidities ?                                                                                                        & None; Diabetes; Heart Disease; Kidney Disease; Gall Stones                                                                                                                                                                                                                                                                              \\ \hline
            15      & Do you have a history of any previous surgeries?                                                                                       & None; Gall Bladder; Intestine; Kidney; Uterus                                                                                                                                                                                                                                                                                           \\ \hline
            16      & Have you experienced any recent changes in your menstrual cycle?                                                                       & Changes in Periods; Absence of Periods; None                                                                                                                                                                                                                                                                                            \\ \hline
            17      & Have you noticed any of these abnormalites?                                                                                            & Abnormal Vaginal Bleeding; Foul Smelling Discharge; None                                                                                                                                                                                                                                                                                \\ \hline
        \end{tabular}
    }
    \caption{Questionnaire for the Diagnostic Protocol.}
    \label{tab:questionnaire}
\end{table}

\noindent The Figure \ref{fig:abdomen_regions_and_patterns} below shows the 9 regions of the abdomen and 4 patterns of pain that are referred to in the questionnaire.
\begin{figure}[H]
    \centering
    \includegraphics[scale=0.08]{images/abdomen_regions_and_patterns.png}
    \caption{Regions of the Abdomen and Patterns of Pain.}
    \label{fig:abdomen_regions_and_patterns}
\end{figure}
%%%%%%%%%%%%%%%%%%%%%%%%%%%%%%%%%%%%%%%%%%%%%%%%%%%%%%%%%%%%%%%%%%
\newpage
\section{Frontend Screenshots for the Implemented CDSS}
\label{appendix:frontend_screenshots}
The following are the screenshots of the frontend of the implemented CDSS. The screenshots are taken from the web application that is developed for the CDSS. The website for the CDSS is hosted at \url{https://kutumlab.github.io/abdominal-pain-cdss/}.\\[\baselineskip]

\noindent \textbf{Android Application}: Four screenshots of the Android application are shown in the Figure \ref{fig:android_app_screenshots}. The screenshots show the question and their respective options.
\begin{figure}[H]
    \centering
    \includegraphics[scale=0.1]{images/android_app_screenshots.png}
    \caption{Screenshots of the Android Application.}
    \label{fig:android_app_screenshots}
\end{figure}

\noindent \textbf{Web Application}: The screenshots of the web application are shown in the Figure \ref{fig:web_app_screenshots}. The screenshots for the same questions and their respective options as in the Android application are shown.
\begin{figure}[H]
    \centering
    \includegraphics[scale=0.05]{images/web_app_screenshots.png}
    \caption{Screenshots of the Web Application.}
    \label{fig:web_app_screenshots}
\end{figure}

\noindent \textbf{Result Page}: The result page of the web application is shown in the Figure \ref{fig:results_page}. The result page shows two screenshots, one with a successful probable and organ of origin diagnosis, and the other with a prompt to visit emergency department, triggered by the presence of danger signs in discriminator section of the questionnaire.
\begin{figure}[H]
    \centering
    \includegraphics[scale=0.07]{images/results_page.png}
    \caption{Result Page of the Web Application.}
    \label{fig:results_page}
\end{figure}

%%%%%%%%%%%%%%%%%%%%%%%%%%%%%%%%%%%%%%%%%%%%%%%%%%%%%%%%%%%%%%%%%%
\section{Additional Sources}
\label{appendix:sources}
The report was compiled using the \href{https://github.com/TUM-LIS/tum-dissertation-latex}{TUM dissertation/PhD thesis} LaTeX template as the base template. Various additional changes were made to the template to suit the requirements of the project report. The template is available under the CC-BY-4.0 license.\\[\baselineskip]

\noindent The Figure \ref{fig:abdominal_regions} (A) was source from Wikimedia Commons and is available \href{https://commons.wikimedia.org/wiki/File:Gray1220-es.svg}{here}. The image of nine abdomen regions in Figure \ref{fig:web_app_screenshots} (B) was sourced from \href{https://www.adam.com/}{A.D.A.M. Consumer Health}.


%-------------------------------------------------------------------------------

\end{document}

%%% Local Variables:
%%% mode: latex
%%% TeX-master: t
%%% End:
