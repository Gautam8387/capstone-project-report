%-------------------------------------------------------------------------------
%
% TUM Dissertation Template
%
% For usage instructions see README.md
%
% Authors:
%   Andre Richter, andre.richter@tum.de
%   Michael Vonbun, michael.vonbun@tum.de
%   Christian Herber, christian.herber@tum.de
%   Stefan Wallentowitz, stefan.wallentowitz@tum.de
%
%-------------------------------------------------------------------------------
\documentclass[%
  % layouttitlepage,            % layout help rules (to see if you need
  %                             % some extra vspace in your title etc.)
  % headings = standardclasses, % serif fonts for headings
  % headings = big,             % If you use serif fonts for headings (above option
  %                             % uncommmented), uncomment this one to get smaller
  %                             % headings
  % sansseriftitlepage,         % sans serif title page
  % notocintoc,                 % do not add toc to toc itself
]{tumDiss}
\usepackage[utf8]{inputenc}

\usepackage[left=30mm,right=30mm,top=30mm,bottom=30mm]{geometry}
\usepackage{fancyhdr}

\pagestyle{fancy}
\fancyhf{} % clear all header and footer fields
\fancyhead[L]{Ahuja, Gautam} % title on left
\fancyhead[R]{\rightmark} % section on right
\fancyfoot[C]{\thepage} % page number in center
\renewcommand{\headrulewidth}{0.4pt}

%-------------------------------------------------------------------------------
% Binding correction for the title page.
% WARNING: ONLY NEEDED FOR THE PRINT VERSION!
%
% After printing and binding, the left part of the titlepage may lose
% significant space, for example due to overlap from glue binding.
% You can increase the left margin of the title page with this option.
% This value of 8mm was measured for glue binding a thesis that was printed by
% the TUM Fachschaft EI and ~140 pages.
%-------------------------------------------------------------------------------
% \titlepagebindingcor{8mm}

%-------------------------------------------------------------------------------
% Binding correction for everything else.
% Does not affect titlePageBindingCor!
%
% WARNING: THIS OPTION CAN SHAKE UP YOUR CURRENT LAYOUT.
% If you want to use it, it is best to work with it from the very start. Adding
% it when finishing your dissertation might get you into trouble.
%
% Search http://texdoc.net/texmf-dist/doc/latex/koma-script/scrguien.pdf for
% "BCOR" for further reading.
%-------------------------------------------------------------------------------
% \KOMAoptions{BCOR=3mm}



%-------------------------------------------------------------------------------
% Faculty
%-------------------------------------------------------------------------------
\faculty{Department of Computer Science}

%-------------------------------------------------------------------------------
% Degree
%-------------------------------------------------------------------------------
% \degree{Doktor-Ingenieurs (Dr.-Ing.)}
\degree{Postgraduate Diploma in Advanced Studies and Research (DipASR)}

%-------------------------------------------------------------------------------
% Title
%
% IMPORTANT:
%
% You must add manual line breaks here. If you don't, you'll get uneven spacing
% between the lines.
% YOU ALSO NEED THE BREAK AT THE LAST LINE.
%-------------------------------------------------------------------------------
\title{%
  Linguistic-agnostic Intelligent Support Systems for\\
   Probable Diagnosis to Aid in Clinical Settings\\
}
% \subtitle{--Extended Titile--}

%-------------------------------------------------------------------------------
% People
%-------------------------------------------------------------------------------
\author{Gautam Ahuja}
\vorsitz{Dr. Anurag Agarwal, Dean, Trivedi School of BioSciences, Ashoka University}
\vorsitzz{Dr. Govind K Makharia, Professor, Department of Gastroenterology and Human Nutrition Adjunct Faculty, AIIMS - New Delhi}
\vorsitzzz{Dr. Rintu Kutum, Faculty Fellow, Department of Computer Science, Ashoka University}
\erstpruef{Dr. Partha Pratim Das, Professor of Computer Science, Ashoka University}

% Use this one for a TUM professor
% \zweitpruef{Prof. Dr.-Ing. Vorname Nachname}

% Or this one for an external professor
\zweitpruef[AIIMS, New Delhi]{Dr. external}

% Optionally, add a third examiner
%\drittpruef[RWTH Aachen]{Prof. Dr.-Ing. Max Mustermann}

%-------------------------------------------------------------------------------
% Hand in date
%
% This is the date of your personal hand-in at the TUM doctoral office.
%-------------------------------------------------------------------------------
\date{01.01.2016}

%-------------------------------------------------------------------------------
% Accepted date
%
% You can set this after your thesis was accepted. For handing in,
% it is not needed (at least for the Electrical Engineering faculty).
%-------------------------------------------------------------------------------
\dateaccepted{10.05.2016}



%-------------------------------------------------------------------------------
% Change language, e.g. to german
%-------------------------------------------------------------------------------
% \usepackage[ngerman]{babel}

%-------------------------------------------------------------------------------
% Compatibility issues
%-------------------------------------------------------------------------------
% If you need pstricks, load it here before everything else.
% Otherwise, tikz patterns won't work
%\usepackage{pstricks}

%-------------------------------------------------------------------------------
% Suggested standard packets are included here
%-------------------------------------------------------------------------------
% Loading scrhack fixes:
%   (1) KOMA-Script incompatible macros used in listings package.
%   (2) Inconsistent anchors in hyperref.
\usepackage{scrhack}


% figure inclusion
\usepackage[
  caption = false,
  font    = footnotesize
]{subfig}
\usepackage{graphicx}
\usepackage{pgfplots}
\usepackage{pgfplotstable}
\tikzset{>=stealth}
\usetikzlibrary{patterns}
\usetikzlibrary{pgfplots.statistics}

% code block insertion
\usepackage{moreverb}
\usepackage{listings}

% math and equations
\usepackage{amsmath}
\usepackage{amssymb}
\usepackage{amsfonts}
\usepackage{upgreek}

% enumeration
\usepackage{enumerate}

% Source code with highlighting
\usepackage{listings}
\lstset{
  basicstyle       = \footnotesize,
  captionpos       = b,
  tabsize          = 4,
  commentstyle     = \color{TUMGreen},
  keywordstyle     = \color{TUMBlue},
  stringstyle      = \color{TUMOrange},
  otherkeywords    = {
    uint64_t,
    uint32_t,
    uint16_t,
    uint8_t,
    u64,
    u32,
    u16,
    u8,
    inline
  },
  numbers          = left,
  xleftmargin      = 7ex,
  aboveskip        = 4ex,
  abovecaptionskip = 2ex,
}

% Support for siunitx
\usepackage{siunitx}
\sisetup{
  exponent-product = \cdot,
  output-product   = \cdot,
  per-mode         = symbol-or-fraction,
  quotient-mode    = fraction,
  binary-units     = true
}

% No widows and orphans
\usepackage[all]{nowidow}

% dummy text
\usepackage{lipsum}

% hyperlinks
% according to its documentation, hyperref should be loaded last
% a list of packages that should be loaded after hyperref can be found at
% https://tex.stackexchange.com/questions/1863/which-packages-should-be-loaded-after-hyperref-instead-of-before
\usepackage{url}
\usepackage[
  hidelinks,
  bookmarksnumbered
]{hyperref}

% If hyperref is used, references to tables and figures link to their captions
% and not the actual tables or figures. This is especially unwanted for figures,
% because their captions are below the figure so that clicking on a link just
% shows the captions and the figure is invisible.
%
% Using the caption package fixes this behaviour.
\usepackage{caption}

% glossary functionality
% loading glossaries after hyperref adds hyperlings to acronyms and glossary
% entries
\usepackage[
  toc,
  acronym,
  style = long
]{glossaries}
\makeglossaries

%-------------------------------------------------------------------------------
% Include custom packages here
%-------------------------------------------------------------------------------

% \usepackage{}


%-------------------------------------------------------------------------------
% TUM CI colors for PGF
%-------------------------------------------------------------------------------
\definecolor{grey60} {RGB} {102, 102, 102} % 60% grey

%-------------------------------------------------------------------------------
% Default values for pgfplots
%-------------------------------------------------------------------------------
\newcommand{\figureHeight}{0.5625} %% 16:9
\pgfplotsset{
  compat           = 1.13,
  grid             = major,
  enlarge x limits = 0,
  cycle list name  = tum,
  major grid style = {dotted},
  minor grid style = {dotted},
  legend style     = {
    at     = {(0.98,0.96)},
    anchor = north east,
  },
  width            = \hsize * 0.9,
  height           = \hsize * 0.9 * \figureHeight,
}

%-------------------------------------------------------------------------------
% Correct bad hyphenation here
%-------------------------------------------------------------------------------
\hyphenation{op-tical net-works semi-conduc-tor}

%-------------------------------------------------------------------------------
% Acronyms (will be sorted alphabetically)
%-------------------------------------------------------------------------------
\newacronym{osllm}{osLLM}{Open Source Large Language Model}
\glsadd{osllm}

\newacronym{samd}{SaMD}{Software as a Medical Device}
\glsadd{samd}

\newacronym{llm}{LLM}{Large Language Model}
\glsadd{llm}

\newacronym{ai}{AI}{Artificial Intelligence}
\glsadd{ai}

\newacronym{nlm}{NLM}{Natural Language Model}
\glsadd{nlm}

\newacronym{cdss}{CDSS}{Clinical Decision Support System}
\glsadd{cdss}



%-------------------------------------------------------------------------------
% Actual document starts here
%-------------------------------------------------------------------------------
\begin{document}
\frontmatter
\maketitle


%-------------------------------------------------------------------------------
\chapter{Acknowledgement}
\lipsum[1]


%-------------------------------------------------------------------------------
\chapter{Certificate}
\lipsum[1]



%-------------------------------------------------------------------------------
\chapter{Abstract}

% \lipsum[1-4]
% This is sample text with a citation \cite{barham2003xen}. Multiple citations \cite{barham2003xen,LIS} can be combined.
% This file is ./abstract.tex
% Citation file is ./references.bib


Using Input
\lipsum[1-4]
This is sample text with a citation \cite{barham2003xen}. Multiple citations \cite{barham2003xen,LIS} can be combined.


%-------------------------------------------------------------------------------
% \chapter{Zusammenfassung}

% \lipsum[1-4]



%-------------------------------------------------------------------------------
\tableofcontents
\listoffigures
\listoftables
\printglossary[type=\acronymtype, nonumberlist]



%-------------------------------------------------------------------------------
\mainmatter
\chapter{Introduction}
\label{chap:introduction}

\lipsum[1-4]


\chapter{Background and Motivation}
\label{chap:bcgmot}
\lipsum[1-4]


\chapter{Literature Survey (State of the Art)}
\label{chap:sota}

Example citations~\cite{barham2003xen, LIS}.
Example acronym usage \gls{osllm}.


\chapter{Content}
\label{chap:content}
\section{Problem Statement}
\lettrine{T}{ }he Department of Gastroenterology and Human Nutrition at AIIMS, New Delhi, handles a high influx of outpatient cases daily. Due to time constraints and the high volume of patients, physicians have limited capacity to conduct comprehensive, protocol-driven evaluations for each patient. For the chief complaint of abdominal pain, a structured diagnostic approach is required. This process typically begins with identifying the pain rating and location, followed by questions on aggravating factors, associated symptoms, and the patient’s medical history. However, given the volume of patients, it becomes difficult for physicians to conduct a complete assessment for every individual.\\[\baselineskip]

\noindent To ensure patient care quality, there is a need for a support system that can streamline the diagnostic process by guiding the patients through initial screening and decreasing the cognitive burden on healthcare providers.\\[\baselineskip]

\noindent The secondary problem is the need for linguistically agnostic systems. Since patients at AIIMS come from diverse linguistic backgrounds, it is essential to create a system that can interact with patients regardless of their language proficiency.

\section{Objectives}
The primary objectives of this project are as follows:
\begin{itemize}
    \item \textcolor{TUMRed}{\textbf{Development of a Department-Specific CDSS}}: To design and develop a \textcolor{TUMBlue}{department-specific}, intelligent \textcolor{TUMBlue}{clinical decision support system} for the screening of \textcolor{TUMBlue}{abdominal pain}.  This system will identify the \textcolor{TUMBlue}{probable diagnosis} and organ of origin based on a structured, protocol-driven questionnaire to aid physicians.
    \item \textcolor{TUMRed}{\textbf{Design of a Linguistic-Agnostic Conversational Agent}}: To design and develop a conversational agent that is linguistic-agnostic to drive above mentioned CDSS.
    \item \textcolor{TUMRed}{\textbf{Deployment and Integration}}: To deploy and integrate the system in the OPD setting at the Department of Gastroenterology and Human Nutrition, AIIMS, New Delhi.
\end{itemize}
The project also has a set of secondary objectives, which are as follows:
\begin{itemize}
    \item \textcolor{TUMRed}{\textbf{Collection of Linguistic Data}}: To collect linguistic data from patients at AIIMS, representing diverse accents, dialects, and language variations. This data will be used to improve the robustness of linguistic-agnostic models for future use.
    \item \textcolor{TUMRed}{\textbf{Evaluation of a Rule-Based and Generative Hybrid System}}: To develop and evaluate a hybrid approach that combines a rule-based, deterministic decision-making engine with a generative AI conversational agent for more robust and explainable clinical support.
\end{itemize}
\section{Scope \& Boundaries}
\subsection{Scope}
\lettrine{T}{ }his project focuses on the development of a department-specific Clinical Decision Support System (CDSS) for the Department of Gastroenterology and Human Nutrition at AIIMS, New Delhi. The primary goal of the system is to support the initial evaluation of patients presenting with abdominal pain.\\[\baselineskip]

The system does not autonomously diagnose patients. Instead, it provides probable diagnoses and suggests a possible organ of origin. Final diagnostic authority remains with the physician, who can audit and modify the system's suggestions based on clinical judgment. It has Autonomous Decision-Making and follows a human-in-the-loop approach.

\subsection{Boundaries}
\lettrine{T}{ }he system is limited to the initial evaluation and probable diagnosis of abdominal pain and does not include physical examination components. It is tailored to the Department of Gastroenterology and Human Nutrition's protocol for abdominal pain at AIIMS, New Delhi, and does not generalize to other medical departments or conditions. Additionally, the system does not make final diagnostic decisions, as the ultimate responsibility for diagnosis rests with the physician.


\chapter{Results \& Discussions}
\label{chap:conclusion}
\lipsum[1-4]


\chapter{Conclusion \& Future Work}
\label{chap:conclusion}
\lipsum[1-4]



%-------------------------------------------------------------------------------
% Bibliography
%-------------------------------------------------------------------------------
\backmatter
% unsrturl-custom works only with hyperref loaded
% use other styles like unsrt when hyperref is not loaded
\bibliographystyle{unsrturl-custom}
\bibliography{src/bibliography}



%-------------------------------------------------------------------------------
% Appendix
%-------------------------------------------------------------------------------
\appendix
\chapter{Appendix}

\lipsum[1-4]

\end{document}

%%% Local Variables:
%%% mode: latex
%%% TeX-master: t
%%% End:
