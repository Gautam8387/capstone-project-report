\section{Background}
% \subsection{Evaluation of Abdominal Pain}
\subsection{Diagnosis of Abdominal Pain in Clinical Setting (OPD)}
\label{sec:abdominal_pain}
\lettrine{A}{ }bdominal pain is one of the most common and diagnostically challenging chief complaints encountered in clinical practice \cite{gans2015guideline}. This serves as a symptom of a wide range of underlying conditions, encompassing both gastrointestinal and non-gastrointestinal issues. This wide range of potential causes increases diagnostic complexity, requiring a systematic and comprehensive evaluation process.\\

\noindent The assessment of abdominal pain typically follows a systematic evaluation strating with a physician-patient interaction. This interaction incorporates multiple clinical dimensions. These dimensions provide clues that aid in narrowing down possible diagnoses. The key aspects considered during an evaluation include:
\begin{itemize}
    \item \textcolor{TUMRed}{\textbf{Location of Pain}}: The abdomen is anatomically divided into nine regions (as shown in Figure \ref{fig:abdominal_regions} (A) below) --- epigastric, umbilical, hypogastric, bilateral hypochondriac, bilateral lumbar, and bilateral iliac regions \cite{AbExm}. The specific region where pain is reported often serves as an essential diagnostic clue \cite{gans2015guideline}.
    \item \textcolor{TUMRed}{\textbf{Presence of Danger Signs}}: Symptoms like lightheadedness, altered sensorium, and respiratory distress may signal critical underlying conditions requiring immediate attention.
    \item \textcolor{TUMRed}{\textbf{Severity of Pain}}: Pain intensity (1-10) is subjectively reported by the patient but often correlates with the urgency of the clinical situation.
    \item \textcolor{TUMRed}{\textbf{Onset of Pain}}: The onset of pain, whether the pain arise suddenly over minutes-hours (acute) or the pain arose gradually over hours-days (insidious).
    \item \textcolor{TUMRed}{\textbf{Character of Pain}}: Pain can be classified as burning, stabbing, pin-pricking, constricting, throbbing, dull aching, or non-specific. Each pain type is linked to a distinct set of possible diagnoses.
    \item \textcolor{TUMRed}{\textbf{Duration of Pain}}: The duration of pain can be classified as either less than 3 months (acute) or more than 3 months (chronic).
    \item \textcolor{TUMRed}{\textbf{Radiation of Pain}}: The direction of pain radiation (to the back, shoulder, groin/inner thigh, arms, or neck) provides vital diagnostic insight.
    \item \textcolor{TUMRed}{\textbf{Aggravating Factors}}: Activities, such as eating, bending forward or sideways, passing stool, passing urine, menstruation, deep inspiration, or walking and exercise, can exacerbate abdominal pain.
    \item \textcolor{TUMRed}{\textbf{Associated Symptoms}}: Symptoms like fever, nausea, vomiting, constipation, diarrhea, jaundice, and changes in bowel habits serve as crucial diagnostic adjuncts. Additionally, systemic symptoms like weight loss, loss of appetite, and signs of anxiety or depression may signal specific gastrointestinal disorders.
    \item \textcolor{TUMRed}{\textbf{Comorbidities}}: Chronic illnesses such as diabetes, cardiovascular disease, kidney disease, or a history of gallstones can affect abdominal pain etiology.
    \item \textcolor{TUMRed}{\textbf{Surgical History}}: Prior surgical procedures on the gallbladder, intestines, kidneys, or uterus can influence the current presentation of abdominal pain.
    \item \textcolor{TUMRed}{\textbf{Gender-Specific Considerations}}: Conditions related to the female reproductive health, such as abnormal vaginal bleeding, menstrual irregularities, and foul discharge, play a role in the evaluation of abdominal pain.
\end{itemize}
\noindent The process of evaluating abdominal pain begins with a physician-patient interaction where history is taken to collect information on the above-mentioned aspects. This is followed by a physical examination, as shown in Figure \ref{fig:abdominal_regions} (B) below, where physicians use visual inspection and hands-on techniques such as palpation, percussion, and auscultation. The examination is often performed with the patient in a supine position with bent knees to relax the abdominal muscles and facilitate assessment \cite{AbExm, cartwright2008evaluation}.
\begin{figure}[H]
    \centering
    \includegraphics[scale=0.09]{images/abdominal_regions_exam.png}
    \caption{A. The nine regions of the abdomen. B. The step-by-step process of a patient undergoing an evaluation of abdominal pain. The dotted red box show where application will be augmented.} 
    \label{fig:abdominal_regions}
\end{figure}

\noindent If the initial evaluation does not provide sufficient diagnostic clarity, physicians may order laboratory tests (e.g. blood work, urine analysis, etc.) or imaging studies (e.g. ultrasound, MRI, or CT scan) to gather further evidence. Based on the collective information from history, physical examination, and diagnostic tests, physicians generate a differential diagnosis --- a list of possible conditions that could explain the symptoms. This process continues until the final diagnosis is reached \cite{ddCleveland}.\\[\baselineskip]

\noindent The application developed as part of this capstone project aims to augment the physician-patient interaction phase of abdominal pain evaluation workflow. By integrating a conversational agent, the system collects essential patient information regarding the dimensions mentioned earlier to identify a probable diagnosis and organ of origin using a deterministic approach. The final decision and subsequent physical examination remain under the physician's control.

\subsection{Brief Introduction to Clinical Decision Support Systems (CDSS)}
\textcolor{TUMBlue}{\textbf{Clinical Decision Making}}\\
\noindent Effective clinical decision-making requires:
\begin{itemize}
    \item Up-to-date Medical Knowledge
    \item Access to Accurate and Complete Patient Data
    \item Good Decision-Making Skills
\end{itemize}
\noindent However, several significant changes have emerged as the healthcare landscape has evolved:
\begin{itemize}
    \item Exponential Growth of Medical Knowledge
    \item Rapid Accumulation of Patient Data
    % \item Increased Complexity in Clinical Inference
    \item Clinical Data Capture and Documentation Burden
\end{itemize}

\noindent Given the changes and the rapid advancements in health, assisting clinicians is becoming more and more important. Thus the role of Clinical Decision Support Systems (CDSS) becomes essential in modern healthcare \cite{visweswaran2022integration}. With the adoption of Artificial Intelligence (AI) in Medicine, existing physicians can handle \textcolor{TUMBlue}{\textbf{more cases}} as systems become \textcolor{TUMBlue}{\textbf{more automated}} \cite{panch2018artificial}.\\[\baselineskip]

\noindent \textcolor{TUMBlue}{\textbf{Clinical Decision Support (CDS)}}\\
\noindent Clinical Decision Support (CDS) refers to a broad set of tools, systems, and interventions aimed at enhancing clinical decision-making. By providing clinicians, healthcare workers, and patients with \textcolor{TUMRed}{\textbf{situation-specific knowledge}}, CDS supports a range of critical healthcare activities such as diagnosis, risk assessment, prognosis, and treatment selection \cite{osheroff2007roadmap}. This support can be delivered through various mediums, including reference guidelines, alerts, and evidence-based recommendations.\\[\baselineskip]

\noindent The goal of CDS is to \textcolor{TUMRed}{\textbf{bridge the gap between clinical knowledge and patient care}}. As medical knowledge expands and healthcare systems become more data-driven, CDS ensures that healthcare providers have access to timely and relevant information. This, in turn, enables more efficient and informed decision-making \cite{osheroff2007roadmap}.\\[\baselineskip]

\noindent \textcolor{TUMBlue}{\textbf{Clinical Decision Support System (CDSS)}}\\
A Clinical Decision Support System (CDSS) is a specialized type of CDS that employs \textcolor{TUMRed}{\textbf{software-based tools}} to directly aid healthcare providers in making clinical decisions. Unlike general CDS, which may include static guidelines or references, CDSS is dynamic, \textcolor{TUMRed}{\textbf{patient-centric}}, interactive and assist physicians. By integrating patient-specific information with a computerized knowledge base, CDSS generates recommendations, alerts, and assessments that assist in diagnosis, treatment, and care planning \cite{hunt1998effects}.

\begin{mdframed}[backgroundcolor=black!10]
    \centering
    \textit{“Any software designed to aid in clinical decision-making by matching patient characteristics to a computerized knowledge base to generate tailored patient-specific assessments or recommendations”}\\
    \flushright
    -  Hunt et al. \cite{hunt1998effects}
\end{mdframed}

\noindent Unlike fully autonomous AI models, CDSS systems are designed to work \textcolor{TUMRed}{\textbf{in collaboration with healthcare professionals}}, offering them interpretive insights and enabling human oversight.\\[\baselineskip]

\noindent \textcolor{TUMBlue}{\textbf{Types of Clinical Decision Support System (CDSS)}}\\
As per \cite{visweswaran2022integration} CDSS can be broadly classified into two categories as shown in Figure \ref{fig:cdss} below:
\begin{itemize}
    \item \textcolor{TUMRed}{\textbf{Knowledge-based CDSS}}: The key components of a knowledge-based AI-CDS system include a knowledge base such as expert-derived rules and an inference mechanism for clinical application such as chained inference for rules.
    \item \textcolor{TUMRed}{\textbf{Data Derived CDSS}}: The key components of a data-derived AI-CDS include a model, such as a data-derived neural network, and an inference mechanism for clinical application, such as forward propagation in a neural network model.
\end{itemize}
\begin{figure}[H]
    \centering
    \includegraphics[scale=0.12]{images/cdss.png}
    \caption{A. Illustrates the principles of decision making process followed by a physician. B. Illustrates the two types of CDSS systems, Knowledge-based and Data-derived.}
    \label{fig:cdss}
\end{figure}

\subsection{Brief Introduction to Conversational AI in Healthcare}
\lettrine{T}{ }he efforts to use natural language system for problem solving using human input can be traced back to the 1960s \cite{bobrow1964natural} and have since evolved significantly. Dialogue systems can be classified into two categories --- task oriented chatbots for specific tasks and open-domain conversational AI for general conversation. Today, conversational AI systems, both mobile and web-based, enbale human-computer interaction using natural, human-like dialogue. Unlike chatbots, which rely on scripted responses to user queries, conversational AI utilizes NLP and LLMs to enable more dynamic, context-aware, and adaptive responses \cite{jurafsky2000speech}. Chatbots follow rigid workflows, while conversational AI systems understand intent, manage multi-turn dialogues, and evolve with user interactions.
\begin{figure}[H]
    \centering
    \includegraphics[scale=0.12]{images/dialogue.png}
    \caption{Types of Dialogue Systems. Task-oriented chatbots such as ELIZA follow a rigid workflow, while conversational AI such as ChatGPT understand intent, manage multi-turn dialogues, and evolve with user interactions.}
    \label{fig:dialogue}
\end{figure}

\noindent The origin of using chatbots in healthcare can be traced back to the development of ELIZA. Joseph Weizenbaum developed one of the first medical chatbots in the 1966 at the Massachusetts Institute of Technology (MIT) \cite{haug2023artificial, weizenbaum1966eliza}. ELIZA employed pattern-matching rules to simulate therapist-like conversations with users. The system was designed to respond to user inputs by reflecting the user's statements back as questions. \\[\baselineskip]

\noindent Since then, Natural Language Processing (NLP) has progressed to the current state of Large Language Models (LLMs). These are advanced \textcolor{TUMRed}{\textbf{AI models}} specifically trained to process, understand, and generate text. This evolution has facilitated the rise of advanced conversational AI systems that can understand, generate, and respond to human language with the help of conversational agents. These agents can understand the context of a conversation, conduct multi-turn dialogue transactions, and establish a more natural interaction with users \cite{clark2019makes}. Since the release of ChatGPT, numerous LLMs---both open-source and proprietary---have been developed at unprecedented speed \cite{clusmann2023future}. These models have been applied to various domains, including healthcare.

\subsection{Brief Introduction to Linguistic-agnostic Systems in Healthcare}
\label{sec:linguistic_agnostic}
\lettrine{C}{ }omputational linguistics focuses on \textcolor{TUMRed}{\textbf{modeling human language}} using computational techniques, enabling machines to understand and process human language. The process of language understanding can be broken down into multiple hierarchical levels, each representing a critical step in the language comprehension pipeline \cite{cohen2022intelligent}. These levels, illustrated in Figure \ref{fig:linguistic_levels}, are as follows:
\begin{itemize}
    \item \textcolor{TUMRed}{\textbf{Phonology}}: This level focuses on sound patterns in language. One of the most notable linguistic tasks at this level is \textcolor{TUMRed}{\textbf{Automatic Speech Recognition (ASR)}}, also known as speech-to-text, where speech waveforms are converted into textual data. The concept of \textcolor{TUMRed}{\textbf{linguistic-agnostic}} systems relates to the ability of ASR models to understand diverse accents, dialects, and speech variations without being restricted to a particular language or pronunciation. For example, a linguistic-agnostic system can process English spoken by people from different regions (e.g., American, British, and Indian accents) with equal accuracy.
    \item \textcolor{TUMRed}{\textbf{Morphology}}: Morphology focuses on how words are formed by combining \textcolor{TUMRed}{\textbf{morphemes}}, the smallest units of meaning. For example, in the word "unhappiness," the morphemes are "un-", "happy", and "-ness".
    \item \textcolor{TUMRed}{\textbf{Syntax}}: Syntax refers to the grammatical structure of language and focuses on how words are arranged in a sentence. It identifies the \textcolor{TUMRed}{\textbf{relationships between words}} and builds a hierarchical representation of the sentence, allowing for proper interpretation
    \item \textcolor{TUMRed}{\textbf{Semantics}}: Semantics deals with \textcolor{TUMRed}{\textbf{word meanings}} and their relationships within sentences. It enables systems to understand the meaning of words, phrases, and sentences after receiving inputs from the phonology, morphology, and syntax layers
    \item \textcolor{TUMRed}{\textbf{Pragmatics}}: This layer focuses on how context influences meaning by going beyond the literal meaning of words. 
    \item \textcolor{TUMRed}{\textbf{Generation}}: While the above layers focus on language understanding, this layer focuses on \textcolor{TUMRed}{\textbf{realistic language generation}} given a computational representation 
\end{itemize}

\begin{figure}[H]
    \centering
    \includegraphics[scale=0.1]{images/linguistic_levels.png}
    \caption{Linguistic Levels}
    \label{fig:linguistic_levels}
\end{figure}

\noindent Accurate interpretation of human language---especially spoken language---is one of the most critical factors that influence the success of \textcolor{TUMRed}{\textbf{human-computer interaction}}. To achieve a \textcolor{TUMRed}{\textbf{linguistic-agnostic system}}, a large, diverse dataset is required to capture regional, phonetic, and dialectal variations in speech.\\[\baselineskip]

\noindent India's linguistic diversity presents a unique challenge for developing linguistic-agnostic systems. With more than 1,652 "mother tongues" and 22 scheduled languages, poses significant challenges for healthcare access across the health literacy spectrum \cite{languagesIndia}. Patients often speak in their native languages and dialects, and may find it challenging to describe symptoms or comprehend medical advice in a language unfamiliar to them. Addressing this challenge requires linguistic-agnostic systems capable of understanding and responding in multiple languages, dialects, and speech patterns.\\[\baselineskip]

\noindent Efforts to bridge linguistic barriers have seen notable contributions within India. The Bhashini initiative \cite{bhashini}, under India's National Language Translation Mission, aims to create linguistic-agnostic systems that allow seamless access to digital services in 22 Indian languages. Using AI-driven voice and text-based translations, Bhashini strives to reduce the language divide across India's vast population. AI4Bharat \cite{ai4bharat}, a research lab at IIT-Madras, advances AI technology for Indian languages, focusing on speech synthesis, automatic speech recognition, and natural language understanding. Their open-source models were trained on diverse Indian languages and dialects data collected from over 400 districts in India with over 15,000 hours of transcribed data, encompassing all 22 scheduled languages of India.\\[\baselineskip]

\noindent The Folk Computing project, with its visions to allow speech input and output to enhance the accessibility of health related information in multiple languages, is a step towards improving accessibility. It is an Android based application with a chat interface powered by LLMs, enabling users to access the health model in multiple languages. The inital version of the application was developed for Hindi, Tamil, Telugu, and Bengali languages at Ashoka University \cite{folkcomp}. A version with Chinese language support was also developed in collaboration with the HealthUnity organization. The demonstration of the application are available on the Folk Computing website \href{https://kutumlab.github.io/folk-comp/#demos}{here}. The complete link to the project is available in Appendix \ref{appendix:links}.

%%%%%%%%%%%%%%%%%%%%%%%%%%%%%%%%%%%%%%%%%%%%%%%%%%%%%%%%%%%%%%%%%%%%%%%%%%%%%%%%%%%%%%%%%%%%%%%%%%%%%%%%%%%%%%%%%%%%%%%%%%%%%%%%%%%%%%%%%%%%%
%%%%%%%%%%%%%%%%%%%%%%%%%%%%%%%%%%%%%%%%%%%%%%%%%%%%%%%%%%%%%%%%%%%%%%%%%%%%%%%%%%%%%%%%%%%%%%%%%%%%%%%%%%%%%%%%%%%%%%%%%%%%%%%%%%%%%%%%%%%%%
\section{Motivation}
\textcolor{TUMBlue}{\textbf{Assisting Physicians in OPD settings}}
\lettrine{T}{ }he \textcolor{TUMRed}{\textbf{Department of Gastroenterology and Human Nutrition}} at the All India Institute of Medical Sciences (AIIMS), New Delhi, was established in 1971 to create qualified gastroenterologists for the country. It is one of the 49 teaching departments and centers at AIIMS, New Delhi.\\[\baselineskip]

\noindent According to the 67th AIIMS Annual Report (2022-2023) \cite{AIIMS2024}, the department managed a total of \textcolor{TUMRed}{\textbf{1,35,944 outpatient department (OPD) cases}}, of which \textcolor{TUMRed}{\textbf{42,586 were new cases}} and \textcolor{TUMRed}{\textbf{93,358 were follow-up cases}}. This is a significant increase from the previous year's figures of 17,790 new cases and 35,622 follow-up cases as shown in Table \ref{tab:aiims_opd} below. In total, the Main Hospital at AIIMS catered to \textcolor{TUMRed}{\textbf{10,39,523 patients}} through its General OPD, Specialty Clinics, and Emergency Department.

% insert table for above data
\begin{table}[h]
    \centering
    \begin{tabular}{|c|c|c|c|}
        \hline
        \textbf{Year} & \textbf{New Cases} & \textbf{Follow-up Cases} & \textbf{Total Cases} \\
        \hline
        2020-2021 & 7,920 & 11,956 & 19,876 \\
        2021-2022 & 17,790 & 35,622 & 53,412 \\
        2022-2023 & 42,586 & 93,358 & 1,35,944 \\
        \hline
    \end{tabular}
    \caption{Number of OPD Cases (new and follow-up) observed at the Department of Gastroenterology and Human Nutrition, AIIMS, New Delhi through the years 2020-2023.}
    \label{tab:aiims_opd}
\end{table}

\noindent The routine gastroenterology OPD operates from Monday to Friday, 8:30 a.m. to 1:00 p.m. \cite{AIIMSOPD}. Given this limited time frame of 270 minutes daily over 5 working days, approximately \textcolor{TUMRed}{\textbf{8,500 new cases are handled per day}}, with multiple physicians addressing various chief complaints. As discussed in Section \ref{sec:abdominal_pain}, one of the most diagnostically challenging chief complaints is \textcolor{TUMRed}{\textbf{abdominal pain}}. A complete protocol of structured questions helps filter the probable diagnosis and the organ of origin. \textcolor{TUMRed}{\textbf{Automating}} this process will significantly assist physicians in managing the high volume of patients the OPD setting.\\[\baselineskip]

\noindent This challenge forms the motivation for the development of an autonomous system. The application aims to augment the physician-patient interaction phase of abdominal evaluation through a conversational agent and assist physicians by generating probable diagnoses and identifying the organ of origin before physical examination. When deployed, the system can handle multiple patients simultaneously via multiple devices. It can conduct the initial evaluation and generate a probable diagnosis and organ of origin. The output will be printed for the patient, who can then present it to the physician for further evaluation.\\[\baselineskip]

\noindent AIIMS also serves as a \textcolor{TUMRed}{\textbf{melting pot for linguistically diverse populations}} from across India. This is reflected in the geographical distribution of inpatients at AIIMS during the year 2021-2022, as shown in the Figure \ref{fig:inpatient_distribution} below \cite{AIIMS2024}. This diversity spans multiple languages, dialects, accents, regional pronunciations, and dialectical variations.

\begin{figure}[H]
    \centering
    \includegraphics[scale=0.13]{images/inpatient_distribution.png}
    \caption{A pie chart showing the geographical distribution of inpatients at AIIMS, New Delhi, during the year 2021-2022. The majority of inpatients come from the states of Delhi, Uttar Pradesh, and Bihar.}
    \label{fig:inpatient_distribution}
\end{figure}

\noindent This diversity in linguistic backgrounds presents a unique opportunity to digitize \textcolor{TUMRed}{\textbf{voice-based}} data related to health from from system-patient interactions. Such digitlization may provide, in the future, valuable insights into the linguistic variations in healthcare settings in India.

%%%%%%%%%%%%%%%%%%%%%%%%%%%%%%%%%%%%%%%%%%%%%%%%%%%%%%%%%%%%%%%%%%%%%%%%%%%%%%%%%%%%%%%%%%%%%%%%%%%%%%%%%%%%%%%%%%%%%%%%%%%%%%%%%%%%%%%%%%%%%
%%%%%%%%%%%%%%%%%%%%%%%%%%%%%%%%%%%%%%%%%%%%%%%%%%%%%%%%%%%%%%%%%%%%%%%%%%%%%%%%%%%%%%%%%%%%%%%%%%%%%%%%%%%%%%%%%%%%%%%%%%%%%%%%%%%%%%%%%%%%%
\section{Problem Statement}
\lettrine{T}{ }he volume of patients at OPD settings related to gastrointestinal complaints is large. Given the time constraints (270 minutes) and the large influx of patients (around 8,500 cases per day), automation and digitlization of the initial physician-patient interaction, specifically for evaluating the chief complaint of abdominal pain, will significantly assist physicians in managing the high volume of patients in the OPD setting.

\section{Objectives}
The objectives are:
\begin{enumerate}
    \item \textcolor{TUMRed}{\textbf{Development of an application}} to assist physicians in the evaluation of abdominal pain in the OPD setting. This application will collect patient responses through a structured, protocol-driven questionnaire and generate a report with a probable diagnosis and organ of origin through a deterministic rule-based system.
    \item \textcolor{TUMRed}{\textbf{Implementing}} a conversational agent that interacts with patients to help collect responses for the questionnaire.
    \item \textcolor{TUMRed}{\textbf{Evaluation}} and assessment of the impact of the rule-based deterministic system with the conversational agent in the OPD setting for abdominal pain evaluation. The evaluation will be performed by comparing no system (baseline), option and click based system with help of a healthcare staff, and a fully conversational system.
\end{enumerate}
