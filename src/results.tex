% Results and Discussion
\section{Results}
% The initial version of Clinical Decision Support System (CDSS) implementation successfully integrated a protocol-driven questionnaire with a linguistic-agnostic conversational agent. The system aims to map patient responses from a 17-question diagnostic questions (categorized into 4 groups) to probable diagnoses and corresponding organs of origin using a deterministic approach for traceable decision-making. The system's compartmentalized architecture --- Android and web-based frontends alongside a containerized backend powered by Google's Gemma 2 9B model, showcases effective modularity and interpretability. 
\lettrine{T}{ }he Clinical Decision Support System (CDSS) implementation successfully addresses the challenge of conducting comprehensive protocol-driven evaluations in high-volume OPD settings. The system integrates a 17-question diagnostic protocol (categorized into discriminators, demographic, female-specific, and general questions) with a linguistic-agnostic conversational agent powered by Google's Gemma 2 9B model. The system's compartmentalized architecture features both Android and web-based frontends alongside a containerized backend, enabling deterministic mapping between patient responses and probable diagnoses with their corresponding organs of origin. This approach ensures transparency and traceability in the decision-making process while maintaining system modularity and interpretability.


\section{Discussion}
% The department-specific approach of the CDSS, as opposed to a general-purpose diagnostic tool, shows promise in addressing the unique needs of AIIMS' Department of Gastroenterology and Human Nutrition. The hybrid approach, combining rule-based decision making with AI-powered conversation handling, offers advantages over purely AI-driven systems by following a human-in-the-loop approach. The system will be evaluated in a clinical setting to assess its impact on patient flow. The successful implementation of text-based interaction lays the groundwork for future voice-based linguistic-agnostic functionality.
\lettrine{T}{ }he department-specific CDSS demonstrates potential in streamlining the systematic evaluation of abdominal pain, which typically requires assessment across multiple clinical dimensions including pain location, intensity, character, duration, and associated symptoms. By automating the initial information-gathering phase, the system supports physicians in following the structured diagnostic framework while managing high patient volumes. The hybrid approach, combining rule-based decision-making with AI-powered conversation handling, maintains human oversight in the diagnostic process - a crucial factor in clinical settings. Unlike autonomous AI systems, this human-in-the-loop approach ensures that final diagnostic decisions remain with physicians while providing them with organized patient responses and probable diagnoses. While the current text-based implementation shows promise, clinical evaluation at AIIMS will assess its impact on patient flow and diagnostic efficiency. The system's foundation also supports future expansion into voice-based linguistic-agnostic functionality, potentially improving accessibility for India's linguistically diverse patient population.