% Results and Discussion
\section{Results}
\lettrine{T}{ }he Clinical Decision Support System (CDSS) implementation successfully addresses the challenge of conducting comprehensive protocol-driven evaluations in high-volume OPD settings. The system integrates a 17-question diagnostic protocol (categorized into discriminators, demographic, female-specific, and general questions) with a linguistic-agnostic conversational agent powered by open source large language models (osLLMs). The system's compartmentalized architecture features both Android and web-based frontends alongside a containerized backend, enabling deterministic mapping between patient responses and probable diagnoses with their corresponding organs of origin. This approach ensures transparency and traceability in the decision-making process while maintaining system modularity and interpretability.


\section{Discussion}
\lettrine{T}{ }he department-specific CDSS demonstrates potential in streamlining the systematic evaluation of abdominal pain, which typically requires assessment across multiple clinical dimensions including pain location, intensity, character, duration, and associated symptoms. By automating the initial information-gathering phase, the system supports physicians in following the structured diagnostic framework while managing high patient volumes. The hybrid approach, combining rule-based decision-making with AI-powered conversation handling, maintains human oversight in the diagnostic process - a crucial factor in clinical settings. Unlike autonomous AI systems, this human-in-the-loop approach ensures that final diagnostic decisions remain with physicians while providing them with organized patient responses and probable diagnoses. While the current text-based implementation shows promise, clinical evaluation at AIIMS will assess its impact on patient flow and diagnostic efficiency. The system's foundation also supports future expansion into voice-based linguistic-agnostic functionality, potentially improving accessibility for India's linguistically diverse patient population.