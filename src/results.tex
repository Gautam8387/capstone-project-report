% Results and Discussion
\section{Results}
\lettrine{T}{ }he Clinical Decision Support System (CDSS) implementation successfully addresses the challenge of conducting comprehensive protocol-driven evaluations in high-volume OPD settings. The system integrates a 17-question diagnostic protocol (categorized into discriminators, demographic, female-specific, and general questions) with a linguistic-agnostic conversational agent powered by open source large language models (osLLMs). The system's compartmentalized architecture features both Android and web-based frontends alongside a containerized backend, enabling deterministic mapping between patient responses and probable diagnoses with their corresponding organs of origin. This approach ensures transparency and traceability in the decision-making process while maintaining system modularity and interpretability.


\section{Discussion}
\lettrine{T}{ }he department-specific CDSS demonstrates potential in streamlining the systematic evaluation of abdominal pain, which typically requires assessment across multiple clinical dimensions including pain location, intensity, character, duration, and associated symptoms. By automating the initial information-gathering phase, the system supports physicians in following the structured diagnostic framework while managing high patient volumes. The hybrid approach, combining rule-based decision-making with AI-powered conversation handling, maintains human oversight in the diagnostic process - a crucial factor in clinical settings. Unlike autonomous AI systems, this human-in-the-loop approach ensures that final diagnostic decisions remain with physicians while providing them with organized patient responses and probable diagnoses. The system's linguistic-agnostic design will allows for easy adaptation to different languages and dialects, potentially enhancing accessibility for patients from diverse linguistic backgrounds. This will also aid the future debelopment of better linguistic-agnostic models for Indian languages. The assessment of the system's impact on evaluation of abdominal pain in OPD settings will provide valuable insights into its diagnostic efficiency and potential for integration into clinical workflows.