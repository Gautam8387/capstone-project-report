% Conclusion and Future Work
\section{Conclusion}
\lettrine{T}{ }he evaluation of abdominal pain remains one of the most diagnostically challenging tasks in clinical practice, requiring systematic assessment across multiple dimensions. In high-volume settings like the Department of Gastroenterology and Human Nutrition at AIIMS, New Delhi, where physicians handle over 1,35,000 OPD cases in 2022-23 alone, conducting comprehensive protocol-driven evaluations for each patient becomes increasingly difficult. While AI-driven Clinical Decision Support Systems offer potential solutions, their black-box nature and autonomous decision-making raise concerns about explainability and accountability in healthcare settings.\\[\baselineskip]

\noindent This project presents an approach that combines the benefits of structured diagnostic protocols with modern AI capabilities. Through collaboration with AIIMS, we developed a department-specific CDSS that uses a deterministic mapping approach to link patient responses from a 17-question diagnostic protocol to probable diagnoses and organs of origin. The system's compartmentalized architecture, featuring both Android and web-based interfaces backed by a containerized backend with open source large language models (osLLMs) powered conversational agent, ensures transparency and traceability in the decision-making process.\\[\baselineskip]

\noindent The system shows how AI can augment the clinical decision-making by maintaining human oversight while streamlining initial patient evaluation. By following a human-in-the-loop approach, the CDSS supports physicians in managing high patient volumes while ensuring that final diagnostic authority remains with healthcare professionals. The successful implementation of the text-based system, designed to handle diverse patient populations at AIIMS, establishes a foundation for more accessible and efficient clinical support tools.

\section{Future Work}
\lettrine{F}{ }uture development of the system will focus on several key areas. First, the Android and web applications will be enhanced with additional features including user registration, authentication, and integration with hospital systems to begin testing. The interface will be optimized based on user feedback from the initial deployment phase.\\[\baselineskip]

\noindent Second future direction is the design and development of the linguistic-agnostic conversational agent's voice-based functionality. This will leverage the diverse patient population at AIIMS to collect data and create robust models capable of handling various Indian languages, accents, and dialects.\\[\baselineskip]

\noindent Another future direction is providing probable diagnoses and organ-of-origin of the symptoms that do not map directly to the current data dictionary. This will involve expanding the system's knowledge base using existing medical ontologies --- such as disease ontologies, UMLS, etc. --- and desiginig a mechanism to provide probable diagnoses for such symptoms.\\[\baselineskip]

\noindent The final phase will involve comprehensive deployment, integration, and evaluation at AIIMS. This will include rigorous testing of the system's impact on patient flow, and diagnostic efficiency. The evaluation will also assess the accuracy of probable diagnoses and organ-of-origin predictions, with feedback from healthcare professionals being incorporated into system improvements.