% Conclusion and Future Work
\section{Conclusion}
\lettrine{T}{ }he evaluation of abdominal pain remains one of the most diagnostically challenging tasks in clinical practice, requiring systematic assessment across multiple dimensions. In high-volume settings like the Department of Gastroenterology and Human Nutrition at AIIMS, New Delhi, where physicians handle over 1,35,000 OPD cases in 2022-23 alone, conducting comprehensive protocol-driven evaluations for each patient becomes increasingly difficult. While AI-driven Clinical Decision Support Systems offer potential solutions, their black-box nature and autonomous decision-making raise concerns about explainability and accountability in healthcare settings.\\[\baselineskip]

\noindent This project presents an approach that combines the benefits of structured diagnostic protocols with modern AI capabilities. Through collaboration with AIIMS, the project aims to develop a department-specific CDSS that uses a deterministic mapping approach to link patient responses from a 17-question diagnostic protocol to probable diagnoses and organs of origin. The system's compartmentalized architecture, featuring both Android and web-based interfaces backed by a containerized backend with open source large language models (osLLMs) powered conversational agent, ensures transparency and traceability in the decision-making process.\\[\baselineskip]

\noindent The system shows how AI can augment the clinical decision-making by maintaining human oversight while streamlining initial patient evaluation. By following a human-in-the-loop approach, the CDSS supports physicians in managing high patient volumes while ensuring that final diagnostic authority remains with healthcare professionals. 
% The successful implementation of the option-based input prototype system, designed to handle diverse patient populations at AIIMS, establishes a foundation for more accessible and efficient clinical support tools.

\section{Future Work}
\lettrine{F}{ }uture development of the system will focus on several key areas. First, the osLLM Agent will be added to the current prototypes of option-based input system to provide a conversational interface for patients. This will involve integrating the osLLM Agent with the existing system to enable natural language conversations with patients. Additionally,the Android and web applications will be enhanced with additional features including voice-based input, and final reports generation.\\[\baselineskip]

\noindent Future future direction is the design and development of the linguistic-agnostic conversational models. This will involve design and development of models leveraging the diverse patient population at AIIMS to create robust models capable of handling various Indian languages, accents, and dialects. \\[\baselineskip]


\noindent Current plan is to encorporate the support of openCHA framework \cite{abbasian2023conversational} for empowering the conversational agent with voice-based functionality and external knowledge support with built-in translation tools as a part of the system. This starting point will be used to develop a more comprehensive linguistic-agnostic conversational agent capable of handling diverse patient populations.\\[\baselineskip]

% \noindent Another future direction is providing probable diagnoses and organ-of-origin of the symptoms that do not map directly to the current data dictionary. This will involve expanding the system's knowledge base using existing medical terminologies, vocabularies, and ontologies. This will include, for example, mapping the abdominal pain codes from ICD-10 to the symptoms from disease ontology and desiginig a mechanism to provide probable diagnoses for such symptoms.\\[\baselineskip]

\noindent The final phase will involve comprehensive deployment, integration, and evaluation at AIIMS. This will include rigorous testing of the system's impact on patient flow, and diagnostic efficiency. The evaluation will also assess the impact of the system on the clinical workflow for abdominal pain evaluation.