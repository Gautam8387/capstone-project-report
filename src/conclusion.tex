% Conclusion and Future Work
\section{Conclusion}
\lettrine{T}{ }he evaluation of abdominal pain remains one of the most diagnostically challenging tasks in clinical practice, requiring systematic assessment across multiple dimensions. In high-volume settings like the Department of Gastroenterology and Human Nutrition at AIIMS, New Delhi, where physicians handle over 1,35,000 OPD cases annually, conducting comprehensive protocol-driven evaluations for each patient becomes increasingly difficult. While AI-driven Clinical Decision Support Systems offer potential solutions, their black-box nature and autonomous decision-making raise concerns about explainability and accountability in healthcare settings.\\[\baselineskip]

\noindent This project presents a novel approach that combines the benefits of structured diagnostic protocols with modern AI capabilities. Through collaboration with AIIMS, we developed a department-specific CDSS that uses a deterministic mapping approach to link patient responses from a 17-question diagnostic protocol to probable diagnoses and organs of origin. The system's compartmentalized architecture, featuring both Android and web-based interfaces backed by a containerized backend with Gemma 2 9B model-powered conversational agent, ensures transparency and traceability in the decision-making process.\\[\baselineskip]

\noindent The system demonstrates how AI can augment rather than replace clinical decision-making by maintaining human oversight while streamlining initial patient evaluation. By following a human-in-the-loop approach, the CDSS supports physicians in managing high patient volumes while ensuring that final diagnostic authority remains with healthcare professionals. The successful implementation of the text-based system, designed to handle diverse patient populations at AIIMS, establishes a foundation for more accessible and efficient clinical support tools.

\section{Future Work}
\lettrine{F}{ }uture development of the system will focus on several key areas. First, the Android and web applications will be enhanced with additional features including user registration, authentication, and integration with hospital information systems to enable seamless patient data management. The interface will be optimized based on user feedback from the initial deployment phase.\\[\baselineskip]

\noindent A crucial advancement will be the development and implementation of the linguistic-agnostic conversational agent's voice-based functionality. This will leverage the diverse patient population at AIIMS to create robust models capable of handling various Indian languages, accents, and dialects. The voice-based system will significantly improve accessibility, particularly for patients with limited literacy or technological familiarity.\\[\baselineskip]

\noindent The final phase will involve comprehensive deployment, integration, and evaluation at AIIMS. This will include rigorous testing of the system's impact on patient flow, diagnostic efficiency, and physician cognitive load. The evaluation will also assess the accuracy of probable diagnoses and organ-of-origin predictions, with feedback from healthcare professionals being incorporated into system improvements. Long-term studies will measure the system's effectiveness in supporting clinical decision-making while maintaining high standards of patient care.

% The immediate focus will be on enhancing the Android and web applications with user registration and authentication features, alongside integration with hospital systems. The primary development effort will concentrate on implementing the linguistic-agnostic conversational agent's voice-based functionality, leveraging AIIMS' diverse patient population to create robust models capable of handling various Indian languages and dialects. Finally, comprehensive deployment and evaluation at AIIMS will assess the system's impact on patient flow, diagnostic efficiency, and physician cognitive load, with continuous improvements based on clinical feedback.