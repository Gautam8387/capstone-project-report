\section{Protocol}
\lettrine{T}{ }he development of the Clinical Decision Support System (CDSS) was guided by a structured protocol provided by the physicians of the Department of Gastroenterology and Human Nutrition at AIIMS, New Delhi. This protocol defines the flow of patient interaction through a set of questions designed to capture essential diagnostic information related to abdominal pain.\\[\baselineskip]

\noindent The initial protocol consisted of a 12-question diagnostic questions that addressed key clinical dimensions of abdominal pain, as described in Section \ref{sec:abdominal_pain}. These questions explored aspects like pain location, intensity, aggravating factors, associated symptoms, etc. However, for a more controlled and streamlined workflow, the questionnaire was later broken down to include 17 refined questions. By refining these questions, a categorization of questions was made possible. For instance, questions related to menstrual cycle health were separated from the general set of symptoms to provide more focused and relevant insights for female patients.\\[\baselineskip]

\noindent Through refinement, the 17 questions were grouped into the following categories:
\begin{enumerate}
    \item \textcolor{TUMRed}{\textbf{Discriminators}}: Critical questions are used to differentiate between emergency cases and regular OPD cases. If the answer to any of these questions indicates a potential emergency, the patient is automatically redirected to the emergency department instead of continuing through the normal OPD process.
    \item \textcolor{TUMRed}{\textbf{Demographic}}: Questions related to the patient's demographic information, such as age and gender.
    \item \textcolor{TUMRed}{\textbf{Female-Specific}}: Questions specific to female health issues --- menstrual health --- that play a significant role in the diagnostic process for women but are irrelevant to male patients.
    \item \textcolor{TUMRed}{\textbf{General}}: These are common questions applicable to all patients, regardless of gender, and form the core of the diagnostic questionnaire.
\end{enumerate}
A detailed breakdown of the 17 questions into the four categories is presented in Figure \ref{fig:question_breakdown}. A tabulated version of the 17 questions and their respective answers can be found in Appendix \ref{appendix:questionnaire}.

\begin{figure}[H]
    \centering
    \includegraphics[scale=0.1]{images/question_breakdown.png}
    \caption{Breakdown of the 17 Questions into Categories}
    \label{fig:question_breakdown}
\end{figure}

\section{Probable Diagnosis and Organ of Origin}
\lettrine{T}{ }he core function of the CDSS is to map patient responses to probable diagnoses and the organ of origin. The process for determining these outcomes relies on a pre-defined list of possible diagnoses and corresponding organs of origin, as provided by the Department of Gastroenterology and Human Nutrition at AIIMS.

\subsection{Diagnosis Mapping}
The physicians provided a list of 29 possible diagnoses related to abdominal pain. Each diagnosis is associated with a specific set of responses to the 17 questions from the protocol. To identify a potential diagnosis, the patient's responses are compared against this set of known answer combinations. If the patient's responses align with one of these pre-defined answer sets, a match is established for the probable diagnosis.

\subsection{Organ-of-Origin Identification}
The CDSS performs the identification of the organ of origin for abdominal pain. This identification is done via the mapping established by linking 29 diagnoses to a set of 19 possible organs of origin. Once a probable diagnosis is identified, the system determines the corresponding organ of origin using a \textcolor{TUMBlue}{deterministic mapping} approach. This allows the system to be interpretable.\\[\baselineskip]

\noindent The relationships between questions, diagnoses, and organs of origin are represented as a graph structure. Each diagnosis is connected to potential answers for the 17 questions, forming distinct paths in the graph. The graphical representation in Figure \ref{fig:graphical_representation} provides a visual explanation of how diagnoses are reached based on patient responses. A tabulated mapping of organs of origin to diagnoses can be found in Appendix \ref{appendix:organs_of_origin}.\\[\baselineskip]

\noindent In Figure \ref{fig:graphical_representation}, the pink nodes (center) correspond to the 29 diagnoses, while the yellow nodes represent the 19 possible organs of origin (center below). The nodes representing answers to the 12 original questions are displayed in 12 distinct colors, each corresponding to a specific question.

\begin{figure}[H]
    \centering
    \includegraphics[scale=0.07]{images/graphical_representation.png}
    \caption{Graphical Representation of the Mapping between Questions, Diagnoses, and Organs of Origin}
    \label{fig:graphical_representation}
\end{figure}

\subsection{Workflow}
The workflow of the questionnaire begins with the patient's interaction with the system through a structured questionnaire. As shown in Figure \ref{fig:workflow}, it starts with discriminator questions to identify emergencies; if flagged, the patient is directed to the emergency department. Otherwise, the patient proceeds to demographic questions to gather basic information regarding age and gender, followed by general questions covering universal symptoms and history. Female patients are asked additional questions related to menstrual health before proceeding to the general questions. The patient's responses are then mapped to probable diagnoses and organs of origin, as described in the previous section. The system generates a report based on the patient's responses, which contains all the answers provided by the patient, the probable diagnosis, and the organ of origin for the physician's reference.
\begin{figure}[H]
    \centering
    \includegraphics[scale=0.1]{images/workflow.png}
    \caption{Workflow of the Patient-Questionnaire Interaction}
    \label{fig:workflow}
\end{figure}

\section{Implementation of the CDSS}
\lettrine{T}{ }he CDSS was implemented using a comprehensive technology stack. The aim was to create a system that is user-friendly, efficient, auditable, and easily accessible to physicians. The system was designed to be integrated into the hospital's existing infrastructure, allowing for seamless interaction with the CDSS.\\[\baselineskip]

\noindent The implementation involved the creation of both an Android application and a web application for patient interaction. The initial version Android application was developed using Java and Kotlin in Android Studio, while the web application was built using Streamlit in Python. Streamlit is an open-source Python framework for data scientists and AI/ML engineers to deliver dynamic data apps \cite{streamlit}. 

\subsection{Frontend Development and User Interface}
The frontend development involved creating a user-friendly interface for patient interaction. The Android application was designed to be intuitive and easy to navigate, with a clean and simple layout. The web application was developed to provide a seamless experience for patients interacting with the system on a desktop or mobile device. The frontend was designed to be responsive, ensuring that the system is accessible across a wide range of devices.\\[\baselineskip]

\noindent The initial prototype of both the Android application was designed using Figma, a web-based design tool. Both applications were designed to follow the workflow defined in Figure \ref{fig:workflow}. Both prototypes followed a click-through option for patient interaction, where the patient could navigate through the questionnaire by answering each question sequentially. The responses were stored in the system and used to generate the probable diagnosis and organ of origin. As seen in \cite{eich1999internet}, even a simple interface with a well-defined workflow can be effective and have a significant impact on user-friendliness.\\[\baselineskip]

\noindent The demonstration of the Android application design can be found on our website \href{https://kutumlab.github.io/abdominal-pain-cdss/#mobile-application}{here}. Another demonstration of the web application for a successful probable diagnosis and organ of origin can be found \href{https://kutumlab.github.io/abdominal-pain-cdss/#web-application-diagnosis}{here}. The demonstration of web application for an emergency department visit can be found \href{https://kutumlab.github.io/abdominal-pain-cdss/#web-application-emergency-visit}{here}. The screenshots and complete URLs of the same are shown in the Appendix \ref{appendix:frontend_screenshots}. A future version of the CDSS will include the interface for voice-based interaction

\subsection{Backend Development}
The backend of the CDSS has two major components: the data dictionary and the conversational agent. The two components work together via a Python script. The entire backend is containerized using Docker for easy deployment and management.

\subsubsection{Data Dictionary}
Multiple data dictionaries contain the mapping between patient responses and probable diagnoses and organs of origin. These dictionaries are used to identify the probable diagnosis and organ of origin based on the patient's responses. Dictionaries are stored in JSON format for easy access and retrieval.
\subsubsection{Conversational Agent}
The conversational agent is responsible for interacting with the patient and guiding them through the questionnaire. The agent asks questions based on the patient's responses and stores the answers for further processing. The design of the conversational agent is shown in Figure \ref{fig:conversational_agent}.

\begin{figure}[H]
    \centering
    \includegraphics[scale=0.15]{images/conversational_agent.png}
    \caption{Creation of Conversational Agent}
    \label{fig:conversational_agent}
\end{figure}

\noindent We create a conversational agent using the open source Large Language Models (osLLMs) like Mistral (\texttt{Mistral 7B})\cite{jiang2023mistral}, Llama (\texttt{Llama 3.1 8B}) \cite{dubey2024llama}, Gemma (\texttt{Gemma 2 9B}) \cite{team2024gemma}, and others. The models were selected based on their performance and parameter size. From the above-mentioned models, \texttt{Gemma 2 9B} was selected for the conversational agent based on its performance and parameter size. We use prompt engineering \cite{liu2023pre} to provide the conversational agent with the necessary context to extract the patient's responses as accurately as possible after asking each question with its respective answer choices. \\[\baselineskip]

\noindent The conversational agent is hosted and managed by using Ollama \cite{ollama}, an open-source platform that allows for the deployment and management of large language models. Ollama provides an efficient environment for running these models locally, ensuring privacy and faster processing. The model is queried with the patient's responses using a REST API. We used Nvidia RTX A5000 GPU for loading the model and processing the patient's responses.

\subsection{System Design}
The system design of the CDSS is shown in Figure \ref{fig:high_level_design}. The system consists of two main components: the frontend and the backend. The frontend includes the Android and web applications, while the backend includes the data dictionary and the conversational agent. The containerized backend can be deployed on a local server or cloud platform such as AWS or Google Cloud.
\begin{figure}[H]
    \centering
    \includegraphics[scale=0.19]{images/high_level_design.png}
    \caption{High-Level Design of the CDSS}
    \label{fig:high_level_design}
\end{figure}

\subsection{Augmented Workflow}
The high-level augmentation of CDSS into the existing workflow of evaluation of abdominal pain is shown in Figure \ref{fig:abdominal_regions} (B), depicted by the dotted box. A more detailed view of the augmented workflow is shown in the Figure \ref{fig:aug_workflow}.
\begin{figure}[H]
    \centering
    \includegraphics[scale=0.12]{images/aug_workflow.png}
    \caption{CDSS Augmented Workflow for Abdominal Pain Evaluation}
    \label{fig:aug_workflow}
\end{figure}

\noindent The patient interacts with the CDSS through the Android or web application, answering a set of structured questions. The responses are processed by the conversational agent, which identifies the probable diagnosis and organ of origin based on the patient's responses. The system generates a report containing the patient's responses, probable diagnosis, and organ of origin. The report is then reviewed by the physician, who uses the information to further the diagnostic process. The auditing of responses and the probable diagnosis is done by the physician to ensure the accuracy of the system. In case of any discrepancies, the physician can ask the patient for additional information to refine the diagnosis.

\subsection{Challenges \& Mitigations}
One of the Challenges of any AI-based CDSS is the issue of explainability. The compartmentalization of the system ensures that the components can be updated independently without affecting the overall functionality of the system. It also ensures that the system can be easily audited and the output of each component can be tracked providing transparency, explainability, and interpretability. In case of any conflicting diagnosis, the output of the system can be easily traced back to the patient's responses and the mapping between the responses and the probable diagnosis.\\[\baselineskip]

\noindent Another challenge is the issue of privacy and security. The system is designed to be deployed on a local server or cloud platform, ensuring that patient data is stored securely and complies with all relevant data protection regulations. The open-source nature of the system allows for easy customization and mitigates the issue of privacy and security. The system can be easily adapted to meet the specific requirements of different departments or hospitals.\\[\baselineskip]

\section{Evaluation}
\lettrine{C}{ }urrently, the CDSS is in the final stages of development and is yet to be deployed for clinical evaluation. The evaluation process will involve a beta testing with a small group of physicians from the Department of Gastroenterology and Human Nutrition at AIIMS. The physicians will interact with the system and provide feedback on the system's usability, accuracy, and overall performance. The feedback will be used to refine the system further before a full-scale deployment. The system will also be evaluated and accessed to measure its impact on the evaluation of abdominal pain in OPD setting.\\[\baselineskip]