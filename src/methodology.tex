% methodology.tex
\lettrine{T}{ }he implementation of an application was done in various steps and several technologies were used for the same. The core component of the application is the protocol-driven questionnaire along with the data for probable diagnoses and organ of origin. The questionnaire and its associated data of probable diagnoses and organ of origin were acquired from the physicians at AIIMS, New Delhi. The questions was then processed, refined and categorized into four categories to be used in the application. The data associated with the probable diagnoses and organ of origin was also processed and refined into a structured format to be used in the application.\\[\baselineskip]

\noindent A step-by-step workflow was designed to implement questionnaire into the application which was developed for two platforms --- mobile (Android based) and web-based. A UI/UX design for frontend and user interface was developed for the mobile application and the web-based application. The mobile application was developed using Android Studio and the web-based application was developed using open-source technologies like Python and Streamlit. A backend was designed to handle the data and the logic of the application. The backend was designed to be modular and easily inspectable. The backend hosts the processed data for probable diagnoses and organ of origin, and the conversational agent for the questionnaire. The backend was deployed on a local server. \\[\baselineskip]

\noindent The final workflow of evaluation of abdominal pain in OPD setting augmented by the above application was also designed. 