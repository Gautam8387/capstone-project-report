\section{Probable Diagnosis and Organ of Origin}
\label{appendix:organs_of_origin}
Given below is a table mapping the 19 organs of origin to the 29 probable diagnoses. The Table \ref{tab:organs_of_origin} is structured such that each row corresponds to a organ of origin, and each value in column \texttt{Probable Diagnoses} corresponds to a (list of) probable diagnosis(es) that can originate from that organ. 
\begin{table}[htbp]
    \centering
    \resizebox{\textwidth}{!}{ % Resize entire table
        \begin{tabular}{|p{1cm}|p{3cm}|p{12cm}|} 
            \hline
            \textbf{No.} & \textbf{Organ of Origin}          & \textbf{Probable Diagnoses}                                                                                                 \\ \hline
            1.               & Gall Bladder, Biliary Ducts       & Biliary Pain                                                                                                                \\ \hline
            2.               & Liver \& Gall Bladder             & Rule Out Hepatobiliary Malignancy                                                                                           \\ \hline
            3.               & Biliary System                    & Acute Cholangitis                                                                                                           \\ \hline
            4.               & Liver                             & \begin{tabular}[c]{@{}l@{}}1. Liver Abscess\\ 2. Acute Hepatitis\end{tabular}                                               \\ \hline
            5.               & Pancreas                          & \begin{tabular}[c]{@{}l@{}}1. Acute Pancreatitis\\ 2. Chronic Pancreatitis\\ 3. Rule Out Pancreatic Malignancy\end{tabular} \\ \hline
            6.               & Stomach                           & Gastritis, Dyspepsia                                                                                                        \\ \hline
            7.               & Heart                             & Cardiac Pain                                                                                                                \\ \hline
            8.               & Spleen                            & Spleen Related                                                                                                              \\ \hline
            9.               & Kidney                            & \begin{tabular}[c]{@{}l@{}}1. Renal/Ureteric Calculi\\ 2. Pyelonephritis\end{tabular}                                       \\ \hline
            10.              & Appendix                          & Appendicitis                                                                                                                \\ \hline
            11.              & Intestine                         & \begin{tabular}[c]{@{}l@{}}1. Diverticulitis\\ 2. Intestinal Obstruction\\ 3. Mesenteric Ischemia\end{tabular}              \\ \hline
            12.              & Disorder of Gut Brain Interaction & Disorder of Gut Brain Interaction                                                                                           \\ \hline
            13.              & Uterus \& Ovary                   & Dysmenorrhoea                                                                                                               \\ \hline
            14.              & Uterus                            & \begin{tabular}[c]{@{}l@{}}1. Pelvic Inflammatory Disease\\ 2. Fibroid\\ 3. Adenomyosis\end{tabular}                        \\ \hline
            15.              & Adnexa                            & \begin{tabular}[c]{@{}l@{}}1. Ectopic Pregnancy\\ 2. Adnexal Lesion\end{tabular}                                            \\ \hline
            16.              & Urinary bladder                   & \begin{tabular}[c]{@{}l@{}}1. Cystitis\\ 2. Bladder stones\end{tabular}                                                     \\ \hline
            17.              & Metabolic                         & Look for Metabolic Causes: Diabetes, Hyperparathyroidism, Lead Intoxication, Porphyria                                      \\ \hline
            18.              & Abdominal Wall                    & Neuropathic Pain (Herpes, etc), Abdominal Wall Hematoma                                                                     \\ \hline
            19.              & Pleural Disorders                  & Pleural Effusion/Pleurodynia                                                                                                \\ \hline
        \end{tabular}
    }
    \caption{Mapping of Organs of Origin to Probable Diagnoses.}
    \label{tab:organs_of_origin}
\end{table}

%%%%%%%%%%%%%%%%%%%%%%%%%%%%%%%%%%%%%%%%%%%%%%%%%%%%%%%%%%%%%%%%%%
\section{Questionnaire}
\label{appendix:questionnaire}
The 17 questions used in the diagnostic questionnaire are listed below along with the possible answers. The Table \ref{tab:questionnaire} is structured such that each row corresponds to a question, and each value in column \texttt{Possible Answers} corresponds to a (list of) possible answer(s) to that question.
\begin{table}[htbp]
    \centering
    \resizebox{\textwidth}{!}{ % Resize entire table
        \begin{tabular}{|p{1cm}|p{5cm}|p{8cm}|}
            \hline
            No. & Question                                                                                                                               & Possible Answers (seperated by semi-colon)                                                                                                                                                                                                                                                                                              \\ \hline
            1       & Please rate the severity of the pain on a scale of 1 to 10                                                                             & 1-3; 4-7; 8-10                                                                                                                                                                                                                                                                                                                          \\ \hline
            2       & Have you experienced any Trauma?                                                                                                       & Yes; No                                                                                                                                                                                                                                                                                                                                 \\ \hline
            3       & Are there any danger signs present?                                                                                                    & Light Headedness; Altered Sensorium; Respiratory Distress; None                                                                                                                                                                                                                                                                         \\ \hline
            4       & Choose your gender                                                                                                                     & Male; Female                                                                                                                                                                                                                                                                                                                            \\ \hline
            5       & Choose your age group                                                                                                                  & 0-15; 15-25; 25-45; 45-60; 60+                                                                                                                                                                                                                                                                                                          \\ \hline
            6       & Where does the pain occur in the abdomen?                                                                                              & \textless{}Image of 9 regions of Abdomen\textgreater{}                                                                                                                                                                                                                                                                                  \\ \hline
            7       & How did the pain start?                                                                                                                & Over Minutes to Hours (Acute); Over Hours to Days (Insidious)                                                                                                                                                                                                                                                                           \\ \hline
            8       & What is the character of your pain?                                                                                                    & Burning Pain; Stabbing Pain; Pin Pricking Pain; Constricting Pain; Throbbing Pain; Dull aching/non-specific Pain                                                                                                                                                                                                                        \\ \hline
            9       & What's the pattern of your pain?                                                                                                       & \textless{}Image of 4 patterns of pain\textgreater{}                                                                                                                                                                                                                                                                                    \\ \hline
            10      & How long have you been experiencing the pain?                                                                                          & Less than 3 months; More than 3 months                                                                                                                                                                                                                                                                                                  \\ \hline
            11      & Does the pain radiate to any other areas?                                                                                              & No Radiation; To Back; To Shoulder; To Groin/Inner Thigh; To Arms/Neck                                                                                                                                                                                                                                                                  \\ \hline
            12      & Does the pain increase or decrease with the following aggravating/relieving factors?                                                   & No Aggravating/Relieving Factors; With Food Intake; Bending Forward; Passing Stool; Passing Urine; Menstruation; Bending Sideways; Deep Inspiration; Walking and Exercise                                                                                                                                                               \\ \hline
            13      & Are there any associated symptoms with your pain? Please specify if you experience none, or if you have any of the following symptoms. & None; Lump; Fever or Chills; Nausea and/or Vomiting; Abdominal Bloating; Constipation; Diarrhoea; Blood in Stools/Black Stools; Jaundice; Burning Micturition; Blood in Urine; Weight Loss; Loss of Appetite; Stress, Anxiety, Depression, Palpitation; Shortness of Breath; Swelling in Neck, Axilla; Skin Changes Over Abdominal Wall \\ \hline
            14      & Do you have any comorbidities ?                                                                                                        & None; Diabetes; Heart Disease; Kidney Disease; Gall Stones                                                                                                                                                                                                                                                                              \\ \hline
            15      & Do you have a history of any previous surgeries?                                                                                       & None; Gall Bladder; Intestine; Kidney; Uterus                                                                                                                                                                                                                                                                                           \\ \hline
            16      & Have you experienced any recent changes in your menstrual cycle?                                                                       & Changes in Periods; Absence of Periods; None                                                                                                                                                                                                                                                                                            \\ \hline
            17      & Have you noticed any of these abnormalites?                                                                                            & Abnormal Vaginal Bleeding; Foul Smelling Discharge; None                                                                                                                                                                                                                                                                                \\ \hline
        \end{tabular}
    }
    \caption{Questionnaire for the Diagnostic Protocol.}
    \label{tab:questionnaire}
\end{table}

\noindent The Figure \ref{fig:abdomen_regions_and_patterns} below shows the 9 regions of the abdomen and 4 patterns of pain that are referred to in the questionnaire.
\begin{figure}[h]
    \centering
    \includegraphics[scale=0.08]{images/abdomen_regions_and_patterns.png}
    \caption{Regions of the Abdomen and Patterns of Pain.}
    \label{fig:abdomen_regions_and_patterns}
\end{figure}
%%%%%%%%%%%%%%%%%%%%%%%%%%%%%%%%%%%%%%%%%%%%%%%%%%%%%%%%%%%%%%%%%%
\newpage
\section{Frontend Screenshots for the Implemented CDSS}
\label{appendix:frontend_screenshots}
The following are the screenshots of the frontend of the implemented CDSS. The screenshots are taken from the web application that is developed for the CDSS. The website for the CDSS is hosted at \url{https://kutumlab.github.io/abdominal-pain-cdss/}.\\[\baselineskip]

\noindent \textbf{Android Application}: Four screenshots of the Android application are shown in the Figure \ref{fig:android_app_screenshots}. The screenshots show the question and their respective options.
\begin{figure}[H]
    \centering
    \includegraphics[scale=0.1]{images/android_app_screenshots.png}
    \caption{Screenshots of the Android Application.}
    \label{fig:android_app_screenshots}
\end{figure}

\noindent \textbf{Web Application}: The screenshots of the web application are shown in the Figure \ref{fig:web_app_screenshots}. The screenshots for the same questions and their respective options as in the Android application are shown.
\begin{figure}[H]
    \centering
    \includegraphics[scale=0.05]{images/web_app_screenshots.png}
    \caption{Screenshots of the Web Application.}
    \label{fig:web_app_screenshots}
\end{figure}

\noindent \textbf{Result Page}: The result page of the web application is shown in the Figure \ref{fig:results_page}. The result page shows two screenshots, one with a successful probable and organ of origin diagnosis, and the other with a prompt to visit emergency department, triggered by the presence of danger signs in discriminator section of the questionnaire.
\begin{figure}[H]
    \centering
    \includegraphics[scale=0.07]{images/results_page.png}
    \caption{Result Page of the Web Application.}
    \label{fig:results_page}
\end{figure}