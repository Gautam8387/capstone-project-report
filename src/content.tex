\section{Problem Statement}
\lettrine{T}{ }he Department of Gastroenterology and Human Nutrition at AIIMS, New Delhi, handles a high influx of outpatient cases daily. Due to time constraints and the high volume of patients, physicians have limited capacity to conduct comprehensive, protocol-driven evaluations for each patient. For the chief complaint of abdominal pain, a structured diagnostic approach is required. This process typically begins with identifying the pain rating and location, followed by questions on aggravating factors, associated symptoms, and the patient’s medical history. However, given the volume of patients, it becomes difficult for physicians to conduct a complete assessment for every individual.\\[\baselineskip]

\noindent To ensure patient care quality, there is a need for a support system that can streamline the diagnostic process by guiding the patients through initial screening and decreasing the cognitive burden on healthcare providers.\\[\baselineskip]

\noindent The secondary problem is the need for linguistically agnostic systems. Since patients at AIIMS come from diverse linguistic backgrounds, it is essential to create a system that can interact with patients regardless of their language proficiency.

\section{Objectives}
The primary objectives of this project are as follows:
\begin{itemize}
    \item \textcolor{TUMRed}{\textbf{Development of a Department-Specific CDSS}}: To design and develop a \textcolor{TUMBlue}{department-specific}, intelligent \textcolor{TUMBlue}{clinical decision support system} for the screening of \textcolor{TUMBlue}{abdominal pain}.  This system will identify the \textcolor{TUMBlue}{probable diagnosis} and organ of origin based on a structured, protocol-driven questionnaire to aid physicians.
    \item \textcolor{TUMRed}{\textbf{Design of a Linguistic-Agnostic Conversational Agent}}: To design and develop a conversational agent that is linguistic-agnostic to drive above mentioned CDSS.
    \item \textcolor{TUMRed}{\textbf{Deployment and Integration}}: To deploy and integrate the system in the OPD setting at the Department of Gastroenterology and Human Nutrition, AIIMS, New Delhi.
\end{itemize}
The project also has a set of secondary objectives, which are as follows:
\begin{itemize}
    \item \textcolor{TUMRed}{\textbf{Collection of Linguistic Data}}: To collect linguistic data from patients at AIIMS, representing diverse accents, dialects, and language variations. This data will be used to improve the robustness of linguistic-agnostic models for future use.
    \item \textcolor{TUMRed}{\textbf{Evaluation of a Rule-Based and Generative Hybrid System}}: To develop and evaluate a hybrid approach that combines a rule-based, deterministic decision-making engine with a generative AI conversational agent for more robust and explainable clinical support.
\end{itemize}
\section{Scope \& Boundaries}
\subsection{Scope}
\lettrine{T}{ }his project focuses on the development of a department-specific Clinical Decision Support System (CDSS) for the Department of Gastroenterology and Human Nutrition at AIIMS, New Delhi. The primary goal of the system is to support the initial evaluation of patients presenting with abdominal pain.\\[\baselineskip]

\noindent The system does not autonomously diagnose patients. Instead, it provides probable diagnoses and suggests a possible organ of origin. Final diagnostic authority remains with the physician, who can audit and modify the system's suggestions based on clinical judgment. It has Autonomous Decision-Making and follows a human-in-the-loop approach.

\subsection{Boundaries}
\lettrine{T}{ }he system is limited to the initial evaluation and probable diagnosis of abdominal pain and does not any other components from abdominal pain evaluation pipeline. It is tailored to the Department of Gastroenterology and Human Nutrition's protocol for abdominal pain at AIIMS, New Delhi, and does not generalize to other medical departments or conditions. Additionally, the system does not make final diagnostic decisions, as the ultimate responsibility for diagnosis rests with the physician.