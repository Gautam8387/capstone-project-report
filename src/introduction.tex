\lettrine{C}{ }omputer-aided clinical decision-making and reasoning have long been considered a model of human behavior. For many years, it has influenced and been the subject of Artificial Intelligence study \cite{cohen2022introducing}. From the very inception of reasoning foundations in medical diagnosis and Clinical Decision Support nearly 65 years ago---where the reasoning was based on symbolic logic and probability understanding and optimum treatment was calculated via value theory---to current advancements in AI, decision systems are being designed to model human knowledge and augment the work of clinicians \cite{ledley1959reasoning, rajkomar2019machine}.\\

\noindent This capstone project, aims to develop an application tailored to address the unique needs in the Outpatient Department (OPD) setting at the Department of Gastroenterology and Human Nutrition, All India Institute of Medical Sciences (AIIMS), New Delhi, specifically for the chief complaint of \textcolor{TUMRed}{\textbf{abdominal pain}}. The application is designed to assist physicians through a linguistic-agnostic conversational agent that engages directly with patients, collecting key responses from a protocol-driven questionnaire. These responses are then processed by a rule-based deterministic system to provide a probable diagnosis and organ of origin.\\

\noindent The remainder of this report is organized as follows:
\begin{itemize}
    \item \textcolor{TUMRed}{\textbf{Chapter 2}} outlines the background and motivation, along with problem statement and objectives of the project.
    \item \textcolor{TUMRed}{\textbf{Chapter 3}} provides a detailed literature survey, identifying the state-of-the-art in CDSS and gaps.
    \item \textcolor{TUMRed}{\textbf{Chapter 4}} explains the methodology, and summarizes the work done.
    \item \textcolor{TUMRed}{\textbf{Chapter 5}} presents work done, results and discussions.
    \item \textcolor{TUMRed}{\textbf{Chapter 6}} concludes the report with future work direction.
\end{itemize}