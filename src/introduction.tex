\lettrine{C}{ }omputer-aided clinical decision-making and reasoning have long been considered a model of human behavior. For many years, it has influenced and been the subject of Artificial Intelligence study \cite{cohen2022introducing}. From the very inception of reasoning foundations in medical diagnosis and Clinical Decision Support nearly 65 years ago---where the reasoning was based on symbolic logic and probability understanding and optimum treatment was calculated via value theory---to current advancements in AI, decision systems are being designed to model human knowledge and augment the work of clinicians \cite{ledley1959reasoning, rajkomar2019machine}.\\

\noindent This project, in collaboration with the Department of Gastroenterology and Human Nutrition at the All India Institute of Medical Sciences, New Delhi, aims to develop a department-specific Clinical Decision Support System (CDSS) tailored to address the unique needs in the Outpatient Department (OPD) setting, specifically for the chief complaint of \textcolor{TUMRed}{\textbf{abdominal pain}}. The system introduces an intelligent CDSS, where a linguistic-agnostic conversational agent engages directly with patients, collecting key responses and extracting information from a protocol-driven questionnaire. This information, along with a probable diagnosis and organ of origin, is then presented to the clinician.\\

\noindent Contemporary AI-based CDSS tends to operate as generalized, multi-purpose diagnostic tools that run autonomously, often assuming full responsibility for diagnosis. By introducing a \textcolor{TUMRed}{\textbf{department-specific}}, \textcolor{TUMRed}{\textbf{protocol-based deterministic approach}} with AI-based linguistical-agnostic conversational agents for providing probable diagnosis and organ of origin, the system ensures transparency, and accountability as the final decision remains with the physician.\\

\noindent The remainder of this report is organized as follows:
\begin{itemize}
    \item \textcolor{TUMRed}{\textbf{Section 2}} outlines the background and motivation, along with problem statement and objectives of the project.
    \item \textcolor{TUMRed}{\textbf{Section 3}} provides a detailed literature survey, identifying the state-of-the-art in CDSS and gaps.
    % \item \textcolor{TUMRed}{\textbf{Section 4}} outlines the problem statement, project objectives, scope, and boundaries of the system.
    \item \textcolor{TUMRed}{\textbf{Section 4}} explains the methodology, system's design and implementation.
    \item \textcolor{TUMRed}{\textbf{Section 5}} presents results and discussions.
    \item \textcolor{TUMRed}{\textbf{Section 6}} concludes the report with future work direction.
\end{itemize}